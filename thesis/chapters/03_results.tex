% !TeX root = ../main.tex
% Add the above to each chapter to make compiling the PDF easier in some editors.

\chapter{Results}\label{ch:results}

\subsection{Intermediate Representation in Relational Algebra}

Introduce operators.

\subsection{Incremental Query Engine}

Tree-walk interpreter that delegates to DBSP. Schema tracking.

Running example: Transitive closure.

\subsection{Datalog Frontend}

Define grammar, outline parser.

\begin{figure}[htpb]
	\centering
	\begin{tabular}{c}
		\begin{lstlisting}[language=EBNF]
		program     = rule* EOF ;
		rule        = head ":-" body "." ;
		head        = IDENTIFIER "(" comparison ( "," comparison )* ")" ;
		body        = ( atom ( "," atom )* )? ;
		atom        = ( "not"? predicate ) | comparison ;
		predicate   = IDENTIFIER "(" IDENTIFIER ( "," IDENTIFIER )* ")" ;

		comparison  = term ( ( "==" | "!=" | ">" | ">=" | "<" | "<=" ) term )? ;
		term        = factor ( ( "+" | "-" ) factor )* ;
		factor      = unary ( ( "*" | "/" ) unary )* ;
		unary       = ( "-" | "!" ) unary | primary ;
		primary     = literal | IDENTIFIER | "(" comparison ")" ;
		literal     = BOOL | UINT | IINT | STRING | NULL ;

		BOOL        = "true" | "false" ;
		UINT        = DIGIT+ ;
		IINT        = ( "-" | "+" )? DIGIT+ ;
		STRING      = "\""<any char except "\"">*"\"" ;
		IDENTIFIER  = ALPHA ( ALPHA | DIGIT )* ;
		ALPHA       = "a".."z" | "A".."Z" | "_" ;
		DIGIT       = "0".."9" ;
		NULL        = "null" ;
        \end{lstlisting}
	\end{tabular}
	\caption[Example listing]{Datalog Grammar of our Variant.}\label{code:datalog-grammar}
\end{figure}

Finding an execution order.

Translating to IR relational algebra.

\subsection{Query Optimization}
