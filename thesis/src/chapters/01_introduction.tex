% !TeX root = ../main.tex
% Add the above to each chapter to make compiling the PDF easier in some editors.

\chapter{Introduction}\label{ch:intro}

\acp{CRDT} are a class of data structures that allow replicas in a distributed
system to converge to a consistent shared state without any coordination
of reads and writes to the data structure, i.e., they can be served from the
local state of a replica.
This allows for offline writes at each replica.
In case of concurrent writes, replicas are guaranteed to converge to the same state,
once they have exchanged their writes with each other.
The convergence guarantee is what renders defining and implementing
correct \acp{CRDT} challenging.
Currently, application developers wanting to use \acp{CRDT} in their applications
face an unfortunate trade-off:
Either they use an existing \ac{CRDT} library~\cite{automerge,yjs,lore}
that is developed by experts in the field using a general-purpose programming language
but exposes a fixed, object-oriented \acs{API} for a fixed set of \acp{CRDT},
or they define their own \acp{CRDT} tailored to their specific use case.
If they opt for the latter, they either have to invest significant efforts
into implementing it and proving that the \ac{CRDT} algorithm
(as well as its implementation) is commutative when applying concurrent writes,
which is a labor-intensive, subtle, and error-prone
task~\cite{gomes2017verifying,kleppmann2022assessing},
requiring expertise in formal verification.
Alternatively, they can learn a \ac{DSL}, like VeriFx~\cite{verifx}
or LoRe~\cite{lore}, which have a restricted expressiveness but allow
\acp{CRDT} to be defined in a way that guarantees their convergence under
concurrent writes.
Yet, restricted \acp{DSL} are not always expressive enough
to define the desired \ac{CRDT}:
For instance, VeriFx can verify a fractional-indexing list \ac{CRDT}
but the verification of the RGA list \ac{CRDT}, which I define in Datalog in
\ref{sec:list-crdt-datalog-dialect}, times out~\cite{verifx}.
Fractional-indexing uses arbitrary precision rational numbers for the list
elements' indices and insertions pick a value between its predecessor and
successor elements as their index.
While this is verifiable, it is inefficient in practice and causes
interleaving issues of concurrent insertions from different replicas~\cite{fugue}.

Previous work~\cite{kleppmann2018data} has proposed Datalog as a \ac{DSL}
to express \acp{CRDT} as queries over a monotonically growing set of
immutable operations.
The set of immutable operations can easily be disseminated among replicas
by gossip, or other protocols such as set
reconciliation~\cite{ratelesssetreconc,rangebasedsetreconc}.
Such protocols are well suited for replication in decentralized peer-to-peer systems.
Crucially, however, is  Datalog's deterministic execution upon sets (or multisets),
which ensure that queries cannot rely on the processing order of operations,
guaranteeing convergence for replicas executing it on the same set of operations,
as I lay out in more detail in \autoref{ch:background}.
With this approach, application developers do not have to worry about
the convergence property of their custom \ac{CRDT} anymore,
as it is guaranteed by construction, and are empowered to design their own \acp{CRDT}.
While still being a restricted language, Datalog may be more expressive
and better known than a custom \ac{DSL}.
Yet, so far the idea has only been hypothesized.
This work aims to put the idea into practice by exploring the feasibility of
expressing \acp{CRDT} as Datalog queries through:

\begin{itemize}
	\item \textbf{Implementing a query engine} which can execute queries
	      represented in an \ac{IR} (\ref{sec:ir}).
	      The \ac{IR} works on top of operators from relational algebra, e.g.,
	      joins, projections, and selections.
	      The engine uses \ac{IVM}, by leveraging the recently released DBSP
	      framework~\cite{budiu2025dbsp}, to incrementally maintain the state
	      of a query over a stream of updates.
	\item \textbf{Defining and implementing a Datalog dialect} (\ref{sec:datalog-frontend}),
	      powerful enough to express \acp{CRDT} as queries.
	      The dialect includes support for recursion and negation
	      and is translated (\ref{sec:datalog-to-relational-algebra})
	      to the query engine's \ac{IR}.
	\item \textbf{Providing \ac{CRDT} definitions} of a \ac{MVR} key-value store
	      (\ref{sec:motivating-example,sec:key-value-stores-datalog-dialect}),
	      a list (\ref{sec:list-crdt-datalog-dialect}), and a causal broadcast
	      (\ref{sec:advanced_example,sec:key-value-stores-datalog-dialect})
	      in my Datalog dialect.
	      \ref{sec:benchmarks} evaluates their performance in two settings:
	      (1) simulating an application restart by feeding in all updates to
	      the \ac{CRDT} at once, and (2) simulating near-real-time collaboration
	      by feeding in recent updates on top of an existing state.
\end{itemize}

Ideally, this different approach to \acp{CRDT} may move them closer to the power,
guarantees and flexibility of database systems.
The higher abstraction level of a query language better supports the decoupling
from logical and physical data representations than object-oriented \acsp{API}.
Additionally, advances in query optimization and performance improvements
of the query engine can be introduced without breaking changes,
something relational databases have benefitted from over the years and
has arguably contributed towards their widespread adoption.
% Maybe add the read finality issue of \acp{CRDT} here and cite with
% \cite{laddad2022keep}. Explain the read finality issue and show how it can
% be addressed with queries.

Conversely, database systems can be introduced to coordination-free environments,
in which every replica can read and write to the database without
coordinating with other replicas.
In exchange for some guarantees, which cannot be upheld in such environments,
e.g. primary key constraints~\cite{bailis2014coordination},
coordination-free database systems can offer ultimate availability:
As long as at least one replica is alive, the database remains available.
This property can be useful in contexts where a system must be resilient
against network partitions.
For instance, in manufacturing, an entire production line should not be stopped
just because some server is not available.
Although \acp{CRDT} are mostly discussed in the context of collaborative
applications, they are not limited to this domain:
Companies like Ditto~\cite{dittoinc} use \acp{CRDT} for applications in
manufacturing, aviation, gastronomy and defense sectors.

\ac{IVM} allows queries to be computed incrementally, i.e., it takes input
\emph{changes} and outputs \emph{changes} to the query's result.
Receiving output changes instead of the full state provides more information
to the application than just the final state of the query.
That way, the idea of functional reactive programming can be extended from
the \acs{GUI} to the whole application stack, including the database layer.
Then, the application can be considered a reactive, pure function
of some base state~\cite{litt2023riffle} and the output changes can be used
to inform the \ac{GUI} which views require rerendering.
Furthermore, showing data changes, e.g. diffs, to the user is naturally supported
without having to (ab)use a database's write-ahead log or similar change
data capture mechanisms.
Most application developers are already used to the concept of a query,
as nearly every application relies on a database in some form.
\ac{CRDT} queries and non-\ac{CRDT} queries can then share the
same interface~\cite{litt2023riffle},
reducing the cognitive load and the complexity of the application stack.

\section{Motivating Example: A Key-Value Store as a Query}\label{sec:motivating-example}

This example demonstrates how a key value store,
consisting of \aclp{MVR}, can be expressed with a query language.
A \ac{MVR} is a generalization of a \ac{LWWR}.
Unlike the latter, a \ac{MVR} exposes conflicting values to the application
as a consequence of concurrent writes to the register.
Therefore, a concurrency detection mechanism is required.
For this example, I use causal histories in which every operation specifies
a set of predecessor operations that it causally depends on.
Version vectors are another mechanism to detect concurrency
but they do not pair well with relational data models.

I use two relations to store the operations on the registers of the key-value store.
The \code{set} relation contains a log of all operations that \emph{set} the
value of a register of the key-value store.
The \code{pred} relation stores the causal dependencies between the operations.
The schema of both relations and some example data is shown in
\ref{fig:mvr-store-pred,fig:mvr-store-set}.
In operation-based \acp{CRDT}, a pair of \code{ReplicaId} and \code{Counter} values
(abbreviated as \code{RepId} and \code{Ctr}, respectively) is frequently used
to uniquely identify an operation.
The \code{ReplicaId} is a unique identifier for the replica that generated the
operation, and the \code{Counter} is essentially a Lamport clock~\cite{lamport2019time}.
Lamport clocks are useful to order concurrent operations on a register.
Having each replica draw a \code{ReplicaId} randomly from a sufficiently large space
such that the probability of a collision is negligible,
this scheme allows replicas to generate unique identifiers without a central authority.

\begin{figure}[htpb]
	\centering
	\small

	\begin{subfigure}[b]{\textwidth}
		\centering
		\begin{subfigure}[b]{0.45\textwidth}
			\centering
			\begin{tabular}{@{}llll@{}}
				\toprule
				RepId   & Ctr    & Key     & Value  \\
				\midrule
				\(r_1\) & 1      & \(k_1\) & \(x\)  \\
				\(r_1\) & 2      & \(k_1\) & \(y\)  \\
				\(r_2\) & 2      & \(k_1\) & \(z\)  \\
				\midrule
				\ldots  & \ldots & \ldots  & \ldots \\
				\midrule
				\(r_1\) & 3      & \(k_2\) & \(a\)  \\
				\(r_2\) & 4      & \(k_2\) & \(b\)  \\
				\(r_2\) & 5      & \(k_2\) & \(c\)  \\
				\bottomrule
			\end{tabular}
			\caption{\code{set} relation}\label{fig:mvr-store-set}
		\end{subfigure}
		\hspace{1em}
		\begin{subfigure}[b]{0.45\textwidth}
			\centering
			\begin{tabular}{@{}llll@{}}
				\toprule
				FromRepId & FromCtr & ToRepId & ToCtr  \\
				\midrule
				\(r_1\)   & 1       & \(r_1\) & 2      \\
				\(r_1\)   & 1       & \(r_2\) & 2      \\
				\midrule
				\ldots    & \ldots  & \ldots  & \ldots \\
				\midrule
				\(r_1\)   & 3       & \(r_2\) & 5      \\
				\(r_2\)   & 4       & \(r_2\) & 5      \\
				\bottomrule
			\end{tabular}
			\caption{\code{pred} relation}\label{fig:mvr-store-pred}
		\end{subfigure}
	\end{subfigure}

	\vspace{1em}

	\begin{subfigure}[b]{\textwidth}
		\centering
		\def\dist{15pt}
		\begin{subfigure}[b]{0.45\textwidth}
			\centering
			\begin{tikzpicture}[]
				% nodes and edges
				\node[op] (k10) {\setop{1}{r_1}{k_1}{x}};
				\node[head,above right=\dist of k10] (k11) {\setop{2}{r_1}{k_1}{y}} edge [pred] (k10);
				\node[head,below right=\dist of k10] (k12) {\setop{2}{r_2}{k_1}{z}} edge [pred] (k10);
			\end{tikzpicture}
			\caption{Causal history of register \(k_1\)}\label{fig:causal-history-k1}
		\end{subfigure}
		\hspace{1em}
		\begin{subfigure}[b]{0.45\textwidth}
			\centering
			\begin{tikzpicture}[]
				\small
				% nodes and edges
				\node[op] (k20) {\setop{3}{r_1}{k_2}{a}};
				\node[head,below right=\dist of k20] (k22) {\setop{5}{r_2}{k_2}{c}} edge [pred] (k20);
				\node[op,below left=\dist of k22] (k21) {\setop{4}{r_2}{k_2}{b}};
				\draw [pred] (k22) edge (k21);
			\end{tikzpicture}
			\caption{Causal history of register \(k_2\)}\label{fig:causal-history-k2}
		\end{subfigure}
	\end{subfigure}

	\caption{
		The relations \code{set} and \code{pred} with example data (top)
		and their causal history illustrated (bottom).
	}\label{fig:mvr-store-relations}
\end{figure}


The causal history of the operations is illustrated on a logical level
in \ref{fig:causal-history-k1,fig:causal-history-k2}.
The edges denote the entries of the \code{pred} relation and a
\setop{Ctr}{RepId}{Key}{Value} label of a node represents a tuple of the
\code{set} relation.
To obtain the state of the key-value store, the following set must be computed:

\begin{align*}
	\var{mvrStore} = \{ (\var{Key}, \var{Value}) \mid
	 & (\var{RepId}, \var{Ctr}, \var{Key}, \var{Value}) \in \var{set}                      \\
	 & \land \nexists (\var{FromRepId}, \var{FromCtr}, \var{\_}, \var{\_}) \in \var{pred}: \\
	 & \var{RepId} = \var{FromRepId} \land \var{Ctr} = \var{FromCtr} \}
\end{align*}

Intuitively, the query selects all key-value pairs from the \code{set} relation
that have not been overwritten.
The result is \(\{ (k_1, v_2), (k_1, v_3), (k_2, u_3)\}\) because other assigned values
(\(v_1\) for \(k_1\); \(u_1, u_2\) for \(k_2\)) have been overwritten by later operations.
\ref{code:mvr-store-datalog} shows a Datalog query that computes the state of the
\ac{MVR} key-value store.
Datalog is build around the concept of deriving new facts from existing ones
by (repeatedly) applying rules which are stated in the form of implications.
Hence, a Datalog query consists of a set of rules, each of which has a
head (left-hand side) and a body (right-hand side), separated by ``:-''.
If the body of a rule is satisfied for an assignment of variables,
the implication is considered true for that assignment,
and a new fact is derived according to the head of the rule.
As Datalog is inspired by first-order logic, the Datalog query of the \ac{MVR}
store resembles the mathematical notation above but uses the
additional \code{overwritten} predicate to make it more readable.

The query can also be expressed in SQL, and I demonstrate two variants.
The first one in \ref{code:mvr-store-sql-left-join} uses a \code{LEFT JOIN}
and a \code{null} filter.
The second one in \ref{code:mvr-store-sql-subquery} uses a subquery
and negative set inclusion, to align the SQL query closer with the mathematical
notation.

\begin{figure}[htpb]
	\centering

	\begin{subfigure}[b]{\textwidth}
		\begin{lstlisting}[keepspaces]
overwritten(RepId, Ctr) :- pred(RepId, Ctr, _, _).
mvrStore(Key, Value)    :- set(RepId, Ctr, Key, Value),
                           not overwritten(RepId, Ctr).\end{lstlisting}
		\caption{The \ac{MVR} key-value store in Datalog.}\label{code:mvr-store-datalog}
	\end{subfigure}

	\vspace{1em}

	\begin{subfigure}[b]{\textwidth}
		\begin{lstlisting}[language=SQL]
SELECT key, value
FROM set LEFT JOIN pred ON set.RepId = pred.FromRepId
                        AND set.Ctr = pred.FromCtr
WHERE pred.FromRepId IS NULL;
        \end{lstlisting}
		\caption{The \ac{MVR} key-value store in SQL using a left join.}\label{code:mvr-store-sql-left-join}
	\end{subfigure}

	\vspace{1em}

	\begin{subfigure}[b]{\textwidth}
		% \begin{tabular}{c}
		\begin{lstlisting}[language=SQL]
WITH overwritten AS (SELECT FromRepId, FromCtr FROM pred)
SELECT key, value FROM set WHERE (RepId, Ctr) NOT IN overwritten;
        \end{lstlisting}
		% \end{tabular}
		\caption{The \ac{MVR} key-value store in SQL using a subquery and set difference.}\label{code:mvr-store-sql-subquery}
	\end{subfigure}

	\caption{The \ac{MVR} key-value store in Datalog and SQL.}\label{code:mvr-store}
\end{figure}

These examples demonstrate how little code is required to express a relatively
simple \ac{CRDT} as a query, as opposed to hand-coding it in an imperative
programming language, demonstrating another advantage of higher level abstractions.
While for this example, the SQL queries are not too far off from the mathematical
notation, the next section shows that this is not always the case and motivates
why I turn towards Datalog as a query language in this work.

\section{Advanced Example: Respecting Causal Order}\label{sec:advanced_example}

\begin{figure}[tpb]
    \centering
    \begin{tikzpicture}
        \small
        \def\dist{15pt}
        \def\shift{+20pt}

        \tikzset{
            tx/.style={draw=gray, dashed},
            tx_label/.style={text=gray, align=left, anchor=north}
        }

        % nodes and edges
        \node[op] (k10) {\setop{1}{r_1}{k_1}{v_1}};

        \node[op,above right=\dist of k10,xshift=\shift] (k11) {\setop{2}{r_1}{k_1}{v_2}}
        edge [pred] (k10);
        \node[op,below right=\dist of k10,xshift=\shift] (k12) {\setop{2}{r_2}{k_1}{v_3}}
        edge [pred] (k10);

        \node[op, below right=\dist of k11,xshift=\shift] (k13) {\setop{3}{r_1}{k_1}{v_4}}
        edge [pred] (k11) edge [pred] (k12);
        \node[op, right=of k13] (k14) {\setop{4}{r_1}{k_1}{v_5}}
        edge [pred] (k13);

        \node[tx,fit=(k10) (k11) (k12)] (tx0) {};
        \node[tx_label] at (tx0.south) {state of origin};
        \node[tx,draw=red,fit=(k13)] (tx1) {};
        \node[tx_label,text=red] at (tx1.south) {\(w_1\) (delayed)};
        \node[tx,fit=(k14)] (tx2) {};
        \node[tx_label] at (tx2.south) {\(w_2\)};

    \end{tikzpicture}
    \caption{
        Example of a causality issue with the naive queries from \autoref{sec:motivating-example}.
    }\label{fig:causal_issue}
\end{figure}


Causal broadcast ensures that each operation is only delivered at each replica
once all of its causal dependencies have been delivered.
The causal dependencies of an operation consist of all previous operations
that may have causally influenced that operation, i.e.,
they are the set of operations whose effects were known to the operation
at the moment it was created.
The (partial) order of operations imposed by a causal broadcast is
called \emph{causal order}.

The example from \autoref{sec:motivating-example} is only a correct \ac{MVR}
\ac{CRDT}, if the tuples in the \code{set} and \code{pred} relations are added in
causal order.
It produces incorrect output if updates are received out-of-order at a replica.
To illustrate this issue, consider \autoref{fig:causal_issue} from the
perspective of replica \(r_2\) and limited to register \(k_1\).
Initially, \(r_2\) is in the familiar state from \autoref{fig:causal-history-k1}.
Then, write \(w_2\) from replica \(r_1\) is delivered to \(r_2\) although its
causal dependency \(w_1\) has not been delivered yet.
At this point, the query from \autoref{sec:motivating-example} would return
\(\{ (k_1, v_2), (k_1, v_3), (k_1, v_5)\} \)
but the correct result respecting causal order is
\(\{ (k_1, v_2), (k_1, v_3) \}\).
\(w_2\) must be buffered and only applied once \(w_1\) has been delivered.

\begin{figure}[tpb]
	\begin{subfigure}[b]{\textwidth}
		% \begin{tabular}{c}
		\begin{lstlisting}[keepspaces]
// EDBPs are omitted in this chapter.
overwritten(RepId, Ctr)     :- pred(RepId, Ctr, _, _).
overwrites(RepId, Ctr)      :- pred(_, _, RepId, Ctr).
isRoot(RepId, Ctr)          :- set(RepId, Ctr, Key, Value),
                               not overwrites(RepId, Ctr).
isLeaf(RepId, Ctr)          :- set(RepId, Ctr, Key, Value),
                               not overwritten(RepId, Ctr).
isCausallyReady(RepId, Ctr) :- isRoot(RepId, Ctr).
isCausallyReady(RepId, Ctr) :- isCausallyReady(FromRepId, FromCtr),
                               pred(FromRepId, FromCtr, RepId, Ctr).
mvrStore(Key, Value)        :- set(RepId, Ctr, Key, Value),
                               isCausallyReady(RepId, Ctr),
                               isLeaf(RepId, Ctr).\end{lstlisting}
		% \end{tabular}
		\caption{The \ac{MVR} key-value store with causal broadcast in Datalog.}\label{code:mvr-crdt-datalog}
	\end{subfigure}

	\vspace{1em}

	\begin{subfigure}[b]{\textwidth}
		% \begin{tabular}{c}
		\begin{lstlisting}[language=SQL]
WITH overwritten AS (SELECT FromRepId, FromCtr FROM pred)
WITH overwrites  AS (SELECT ToRepId, ToCtr FROM pred)
WITH isRoot      AS (SELECT RepId, Ctr FROM set
                    WHERE (RepId, Ctr) NOT IN overwrites)
WITH isLeaf      AS (SELECT RepId, Ctr FROM set
                    WHERE (RepId, Ctr) NOT IN overwritten)
WITH RECURSIVE isCausallyReady AS (
    SELECT * FROM isRoot
    UNION [ALL]
    SELECT pred.ToRepId, pred.ToCtr
    FROM pred, isCausallyReady
    WHERE pred.FromRepId = isCausallyReady.RepId
    AND pred.FromCtr = isCausallyReady.Ctr
)

SELECT set.key, set.value
FROM set, isCausallyReady, isLeaf
WHERE set.RepId = isCausallyReady.RepId
AND set.Ctr = isCausallyReady.Ctr
AND isCausallyReady.RepId = isLeaf.RepId
AND isCausallyReady.Ctr = isLeaf.Ctr\end{lstlisting}
		% \end{tabular}
		\caption{The \ac{MVR} key-value store with causal broadcast in SQL.}\label{code:mvr-crdt-sql}
	\end{subfigure}

	\caption{The \ac{MVR} key-value store with a causal broadcast in Datalog and SQL.}\label{code:mvr-crdt}
\end{figure}

To prevent this, the query has to detect such ``gaps'' in the causal history.
Hence, the problem of causal delivery is equivalent to a graph reachability
problem: Which nodes are reachable from the set of root nodes, i.e., the set of
nodes that are not causally dependent on any other operation.
\ref{code:mvr-crdt-datalog} extends the query from \autoref{sec:motivating-example}
with a causal broadcast expressed in Datalog.
It additionally introduces the four predicates, \code{overwrites}, \code{isRoot},
\code{isLeaf}, and \code{isCausallyReady}, to only consider operations
which are causally ready, to derive the state of the key-value store.
The \code{overwrites} predicate is analogous to the \code{overwritten} predicate.
The \code{isRoot} predicate captures the root nodes of the causal history,
and the \code{isLeaf} predicate captures the leaf nodes of the causal history.
Leaf nodes are the set of nodes that are not (yet) overwritten by other operations.
The \code{isCausallyReady} predicate is defined recursively and captures
the transitive closure of the \code{pred} relation,
if starting from the root nodes of the causal history.
I return to the computation of the transitive closure in \ref{ch:background} to
explain the semantics of Datalog.
Moreover, I use the \ac{MVR} \ac{CRDT} Datalog query throughout this work
as an ongoing example because of its interesting properties:
It is a simple yet realistic example which makes use of recursion and negation,
two features that I deem as important for expressing \acp{CRDT}.

While SQL has the advantage of being more widely known,
Datalog's syntax excels at concisely expressing recursion and
composition~\cite{abo2024convergence}:
\ref{code:mvr-crdt-sql} shows an equivalent SQL query which I structure
with the same subqueries as in the Datalog example,
but the SQL query is more verbose and arguably less readable.
Especially, I want to avoid the cumbersome syntax around recursive
common table expressions which remains unpopular even within the SQL
community~\cite{neumann2024critique, hirn2023fix, mcsherry2022recursion}.
The semantics of recursion in both Datalog and SQL are based on least
fixed point iteration, thereby sharing the same limitations, such as requiring
monotonicity of the computation to guarantee the fixed point's existence
and uniqueness. This precludes the use of negation in recursive queries,
as I elaborate on in \ref{sec:datalog-negation} when introducing stratified
negation.
Hirn et al. \cite{hirn2023fix} discuss these limitations in more detail
in the context of SQL and propose alternatives.
Beyond this, SQL is based on multisets and Datalog on sets but Datalog can
be extended to accommodate multisets as well, as I do when defining
my Datalog dialect in \ref{sec:datalog-frontend}.
While SQL offers support for aggregates,
Datalog requires an extension to support aggregates~\cite{green2013datalog}.
Yet, aggregates are not required to express the \acp{CRDT} of this work.
Hence, there is no fundamental difference for my use case
in terms of their semantics and expressiveness.
Therefore, I prefer Datalog because of its syntax which allows for compact
and composable queries. Composability is key to enabling reusability.
Furthermore, there exists some recent research about defining Datalog
over arbitrary semirings~\cite{abo2024convergence, khamis2022datalog},
which may open up new avenues for Datalog's expressiveness in the future.

The example also demonstrates why atomic writes to the database are important.
Write \(w_2\) updates both the \code{set} and \code{pred} relations.
If the query reads state in which only the \code{set} relation,
but not the \code{pred} relation, has been updated,
the query incorrectly deems \setop{4}{r_1}{k_1}{v_5} as a new root and
again returns the invalid result from above.

While the issue of this section can be ignored by assuming a causal broadcast
either on the application or on the database layer,
I think that queries benefit from having the full causal history available.
It provides them with the ability to detect when operations are concurrent,
and use that information to adjust their conflict handling to perform
custom resolution logic.

\section{Query Execution}

\acp{CRDT} are often used in a near-real-time collaboration settings which
define the access patterns the query engine has to deal with.
Near-real-time collaboration results in frequent updates both from the local
and remote replicas and each update triggers a (re)evaluation of the
\ac{CRDT} query at each replica.
Each individual update is usually small compared to the total size of the
causal history.
Only if a user has been working offline, they may send a larger batch of
updates when they come back online.
To aggravate the issue, operation-based \acp{CRDT} derive their state from a
\emph{monotonically growing} set of operations.

This is a poor fit for traditional query engines, which are stateless,
meaning that they ``forget'' any work done for previous evaluations of the query,
and start from scratch every time.
With a monotonically growing set of operations, this implies a linear
growth of the time it takes to evaluate the query, implying a threshold
upon which the query engine is no longer able to evaluate the query within
an acceptable time budget.

To address this issue, I focus on exploring \ac{IVM} for this work.
The promise of \ac{IVM} is that it can incrementally maintain a view defined
by a query, while only doing work relative to the size of the query's
inputs' \emph{change since the last evaluation}, as opposed to doing work
relative to the size of the query's \emph{full inputs},
as happening with traditional query engines.
This promise may render query evaluations independent of the size
(and correlatedly age) of the causal history but only dependent on the
size of the accrued updates since the last evaluation.
