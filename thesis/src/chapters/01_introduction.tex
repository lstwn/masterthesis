% !TeX root = ../main.tex
% Add the above to each chapter to make compiling the PDF easier in some editors.

\chapter{Introduction}\label{ch:intro}

Today's \ac{CRDT} implementations are built around the in-memory,
object-oriented paradigm~\cite{laddad2022keep} in which there is an API
in the application language with some predefined interface that
specifies how to update and read the state of a \ac{CRDT}.
However, there is also the idea of defining \acp{CRDT}
as relational queries~\cite{kleppmann2018data}.
In this alternative model, the state of a CRDT is obtained through a query over
relations storing operations which are gossiped across replicas.
This research aims to explore this alternative model further and
to utilize Datalog as a potential query language due to its elegance in
expressing recursive queries, which I believe to be a fundamental aspect in
successfully defining \acp{CRDT} as queries.
Furthermore, the declarative nature of a query language may be easier to work
with than lower-level, in-memory, object-oriented data structures.
For this chapter I assume familiarity with \acp{CRDT} and Datalog.
The topics are explained in more detail in \autoref{ch:background}.

\section{Motivating Example: A Key-Value Store as a Query}\label{sec:motivating-example}

A \ac{MVR} is a generalization of a \ac{LWWR}.
Unlike the latter, a \ac{MVR} exposes conflicting values to the application
as a consequence of concurrent writes to the register.
Therefore, a concurrency detection mechanism is required.
For this example, I use causal histories in which every operation specifies
a set of predecessor operations on which it causally depends on\footnotemark{}.
In its totality, this example demonstrates how a key value store,
consisting of MVR registers, can be expressed with a query language.

\footnotetext{
	Version vectors are another mechanism to detect concurrency
	but they do not play as nice with relational data models.
}

I use two relations (\acp{EDBP} in Datalog terms) to store the operations
on the registers of the key-value store.
First, the \code{set} relation can be thought of a log of all operations
that ever happen to the key-value store.
Second, the \code{pred} relation stores the causal dependencies between operations.
The schema of both relations and some example data is shown in
\ref{fig:mvr-store-pred,fig:mvr-store-set}.
In operation-based \acp{CRDT}, a pair of \code{ReplicaId} and \code{Counter} values
(abbreviated as \code{RepId} and \code{Ctr}, respectively) uniquely
identifies an operation.
The \code{ReplicaId} is a unique identifier for the replica that performed the
operation, and the \code{Counter} is essentially a Lamport clock~\cite{lamport2019time}.

\begin{figure}[htpb]
	\centering
	\small

	\begin{subfigure}[b]{\textwidth}
		\centering
		\begin{subfigure}[b]{0.45\textwidth}
			\centering
			\begin{tabular}{@{}llll@{}}
				\toprule
				RepId   & Ctr    & Key     & Value  \\
				\midrule
				\(r_1\) & 1      & \(k_1\) & \(x\)  \\
				\(r_1\) & 2      & \(k_1\) & \(y\)  \\
				\(r_2\) & 2      & \(k_1\) & \(z\)  \\
				\midrule
				\ldots  & \ldots & \ldots  & \ldots \\
				\midrule
				\(r_1\) & 3      & \(k_2\) & \(a\)  \\
				\(r_2\) & 4      & \(k_2\) & \(b\)  \\
				\(r_2\) & 5      & \(k_2\) & \(c\)  \\
				\bottomrule
			\end{tabular}
			\caption{\code{set} relation}\label{fig:mvr-store-set}
		\end{subfigure}
		\hspace{1em}
		\begin{subfigure}[b]{0.45\textwidth}
			\centering
			\begin{tabular}{@{}llll@{}}
				\toprule
				FromRepId & FromCtr & ToRepId & ToCtr  \\
				\midrule
				\(r_1\)   & 1       & \(r_1\) & 2      \\
				\(r_1\)   & 1       & \(r_2\) & 2      \\
				\midrule
				\ldots    & \ldots  & \ldots  & \ldots \\
				\midrule
				\(r_1\)   & 3       & \(r_2\) & 5      \\
				\(r_2\)   & 4       & \(r_2\) & 5      \\
				\bottomrule
			\end{tabular}
			\caption{\code{pred} relation}\label{fig:mvr-store-pred}
		\end{subfigure}
	\end{subfigure}

	\vspace{1em}

	\begin{subfigure}[b]{\textwidth}
		\centering
		\def\dist{15pt}
		\begin{subfigure}[b]{0.45\textwidth}
			\centering
			\begin{tikzpicture}[]
				% nodes and edges
				\node[op] (k10) {\setop{1}{r_1}{k_1}{x}};
				\node[head,above right=\dist of k10] (k11) {\setop{2}{r_1}{k_1}{y}} edge [pred] (k10);
				\node[head,below right=\dist of k10] (k12) {\setop{2}{r_2}{k_1}{z}} edge [pred] (k10);
			\end{tikzpicture}
			\caption{Causal history of register \(k_1\)}\label{fig:causal-history-k1}
		\end{subfigure}
		\hspace{1em}
		\begin{subfigure}[b]{0.45\textwidth}
			\centering
			\begin{tikzpicture}[]
				\small
				% nodes and edges
				\node[op] (k20) {\setop{3}{r_1}{k_2}{a}};
				\node[head,below right=\dist of k20] (k22) {\setop{5}{r_2}{k_2}{c}} edge [pred] (k20);
				\node[op,below left=\dist of k22] (k21) {\setop{4}{r_2}{k_2}{b}};
				\draw [pred] (k22) edge (k21);
			\end{tikzpicture}
			\caption{Causal history of register \(k_2\)}\label{fig:causal-history-k2}
		\end{subfigure}
	\end{subfigure}

	\caption{
		The relations \code{set} and \code{pred} with example data (top)
		and their causal history illustrated (bottom).
	}\label{fig:mvr-store-relations}
\end{figure}


The causal history of the operations is illustrated on a logical level
in \ref{fig:causal-history-k1,fig:causal-history-k2}.
The edges represent the entries of the \code{pred} relation and a node's
\setop{Counter}{ReplicaId}{Key}{Value} label denotes a tuple of the \code{set}
relation.
To obtain the state of the key value store, the following set must be computed:

\begin{align*}
	\var{mvrStore} = \{ (\var{Key}, \var{Value}) \mid
	 & (\var{RepId}, \var{Ctr}, \var{Key}, \var{Value}) \in \var{set}                      \\
	 & \land \nexists (\var{FromRepId}, \var{FromCtr}, \var{\_}, \var{\_}) \in \var{pred}: \\
	 & \var{RepId} = \var{FromRepId} \land \var{Ctr} = \var{FromCtr} \}
\end{align*}

Intuitively, the query selects all key-value pairs from the \code{set} relation
that have not been overwritten.
The result is \(\{ (k_1, y), (k_1, z), (k_2, c)\}\) because other assigned values
(\(x\) for \(k_1\); \(a, b\) for \(k_2\)) have been overwritten by later operations.
\ref{code:mvr-store-datalog} shows a Datalog query that computes the state of the
\ac{MVR} key-value store.
Its formulation is close to the mathematical notation above.
The query can also be expressed in SQL, and I demonstrate two variants.
The first one in \ref{code:mvr-store-sql-left-join} uses a \code{LEFT JOIN}
and a \code{null} filter.
The second one in \ref{code:mvr-store-sql-subquery} uses a subquery
and negated set inclusion, to align the SQL query closer with the mathematical
notation.

\begin{figure}[htpb]
	\centering

	\begin{subfigure}[b]{\textwidth}
		% \begin{tabular}{c}
		\begin{lstlisting}[keepspaces]
overwritten(RepId, Ctr) :- pred(RepId, Ctr, _, _).
mvrStore(Key, Value)    :- set(RepId, Ctr, Key, Value),
                           not overwritten(RepId, Ctr).\end{lstlisting}
		% \end{tabular}
		\caption{The \ac{MVR} key-value store in Datalog.}\label{code:mvr-store-datalog}
	\end{subfigure}

	\vspace{1em}

	\begin{subfigure}[b]{\textwidth}
		% \begin{tabular}{c}
		\begin{lstlisting}[language=SQL]
SELECT key, value
FROM set LEFT JOIN pred ON set.RepId = pred.FromRepId
                        AND set.Ctr = pred.FromCtr
WHERE pred.FromRepId IS NULL;
        \end{lstlisting}
		% \end{tabular}
		\caption{The \ac{MVR} key-value store in SQL using a left join.}\label{code:mvr-store-sql-left-join}
	\end{subfigure}

	\vspace{1em}

	\begin{subfigure}[b]{\textwidth}
		% \begin{tabular}{c}
		\begin{lstlisting}[language=SQL]
WITH overwritten AS (SELECT FromRepId, FromCtr FROM pred)
SELECT key, value FROM set WHERE (RepId, Ctr) NOT IN overwritten;
        \end{lstlisting}
		% \end{tabular}
		\caption{The \ac{MVR} key-value store in SQL using a subquery and set difference.}\label{code:mvr-store-sql-subquery}
	\end{subfigure}
\end{figure}

While for this example the SQL queries are not too far off from the mathematical
notation, I think that Datalog is more elegant, offers better support for
composition, and excels at expressing recursion~\cite{abo2024convergence},
which is important for more complex CRDTs~\cite{kleppmann2018data}.
Especially, I want to avoid the cumbersome syntax around recursive
CTEs which remains unpopular even within the SQL
community~\cite{neumann2024critique, hirn2023fix, mcsherry2022recursion}.
Furthermore, there exists some interesting research about defining Datalog
over arbitrary semirings~\cite{abo2024convergence, khamis2022datalog},
which may open up new avenues for Datalog's semantics and expressiveness.
Finally, I emphasize that recursion in SQL also suffers from requiring
monotonically growing sets for recursive queries~\cite{hirn2023fix},
just like recursion in Datalog.
Hence, there is no fundamental difference in terms of their semantics and
expressiveness but I see big gap in their syntax choices.

\section{Advanced Example: Respecting Causal Order}\label{sec:advanced_example}

\begin{figure}[tpb]
    \centering
    \begin{tikzpicture}
        \small
        \def\dist{15pt}
        \def\shift{+20pt}

        \tikzset{
            tx/.style={draw=gray, dashed},
            tx_label/.style={text=gray, align=left, anchor=north}
        }

        % nodes and edges
        \node[op] (k10) {\setop{1}{r_1}{k_1}{v_1}};

        \node[op,above right=\dist of k10,xshift=\shift] (k11) {\setop{2}{r_1}{k_1}{v_2}}
        edge [pred] (k10);
        \node[op,below right=\dist of k10,xshift=\shift] (k12) {\setop{2}{r_2}{k_1}{v_3}}
        edge [pred] (k10);

        \node[op, below right=\dist of k11,xshift=\shift] (k13) {\setop{3}{r_1}{k_1}{v_4}}
        edge [pred] (k11) edge [pred] (k12);
        \node[op, right=of k13] (k14) {\setop{4}{r_1}{k_1}{v_5}}
        edge [pred] (k13);

        \node[tx,fit=(k10) (k11) (k12)] (tx0) {};
        \node[tx_label] at (tx0.south) {state of origin};
        \node[tx,draw=red,fit=(k13)] (tx1) {};
        \node[tx_label,text=red] at (tx1.south) {\(w_1\) (delayed)};
        \node[tx,fit=(k14)] (tx2) {};
        \node[tx_label] at (tx2.south) {\(w_2\)};

    \end{tikzpicture}
    \caption{
        Example of a causality issue with the naive queries from \autoref{sec:motivating-example}.
    }\label{fig:causal_issue}
\end{figure}


Without the assumption of a causal broadcast,
the example from \autoref{sec:motivating-example} is not a proper \ac{MVR} \ac{CRDT},
as it does not respect the causal order of updates in case updates are delivered
out-of-order.
To illustrate this issue, consider \autoref{fig:causal_issue} from the
perspective of replica \(r_2\) limited to register \(k_1\).
Initially, \(r_2\) is in the familiar state from \autoref{fig:causal-history-k1}.
Then, write \(w_2\) from replica \(r_1\) is delivered to \(r_2\) although its
causal dependency \(w_1\) has not been delivered yet.
At this point, the query from \autoref{sec:motivating-example} would return
\(\{ (k_1, y_1), (k_1, z), (k_1, y_3)\} \)
but the correct result respecting causal order is
\(\{ (k_1, y_1), (k_1, z) \}\).
The value \(y_3\) is delivered too early by eagerly applying the write \(w_2\)
without awaiting its causal dependency \(w_1\) first.
If \(w_1\) is eventually delivered, the query ``jumps'' to the result
\( \{ (k_1, y_3) \} \), skipping the intermediate state
that overwrites the conflicting values \(y_1\) and \(z\) with \(y_2\).
This falsely suggests that the value \(y_3\) suddenly overwrote its previously
reported siblings \(y_1\) and \(z\).

To prevent this, the query has to detect such ``gaps'' in the causal history.
Hence, the problem of causal delivery is equivalent to a graph reachability
problem: Which nodes are reachable from the set of root nodes?
\ref{code:mvr-crdt-datalog} extends the query from \autoref{sec:motivating-example}
with a causal broadcast expressed in Datalog.
It additionally introduces the predicates \code{overwrittes}, \code{isRoot},
\code{isLeaf}, and \code{isCausallyReady} to only consider operations
which are causally ready to derive the state of the key-value store.
The \code{isCausallyReady} predicate is defined recursively and captures
the transitive closure of the \code{pred} relation if starting from the
root nodes of the causal history.
The computation of the transitive closure is used in \ref{ch:background} to
explain the semantics of Datalog in greater detail.
Moreover, I use the \ac{MVR} \ac{CRDT} Datalog query throughout the thesis
as an ongoing example because of its interesting properties:
It is a simple yet expressive example that encompasses recursion and negation,
two features that are important for expressing \acp{CRDT}.

\ref{code:mvr-crdt-sql} shows an equivalent SQL query.
Although I structure the query with the same subqueries as in the Datalog
query, the SQL query is more verbose and arguably less readable.
Hence, I prefer Datalog's syntax which allows for declarative, compact
and composable queries. Ergonomic composability is key to enabling reusability.

\begin{figure}[tpb]
	\begin{subfigure}[b]{\textwidth}
		% \begin{tabular}{c}
		\begin{lstlisting}[keepspaces]
// EDBPs are omitted in this chapter.
overwritten(RepId, Ctr)     :- pred(RepId, Ctr, _, _).
overwrites(RepId, Ctr)      :- pred(_, _, RepId, Ctr).
isRoot(RepId, Ctr)          :- set(RepId, Ctr, Key, Value),
                               not overwrites(RepId, Ctr).
isLeaf(RepId, Ctr)          :- set(RepId, Ctr, Key, Value),
                               not overwritten(RepId, Ctr).
isCausallyReady(RepId, Ctr) :- isRoot(RepId, Ctr).
isCausallyReady(RepId, Ctr) :- isCausallyReady(FromRepId, FromCtr),
                               pred(FromRepId, FromCtr, RepId, Ctr).
mvrStore(Key, Value)        :- set(RepId, Ctr, Key, Value),
                               isCausallyReady(RepId, Ctr),
                               isLeaf(RepId, Ctr).\end{lstlisting}
		% \end{tabular}
		\caption{The \ac{MVR} key-value store with causal broadcast in Datalog.}\label{code:mvr-crdt-datalog}
	\end{subfigure}

	\vspace{1em}

	\begin{subfigure}[b]{\textwidth}
		% \begin{tabular}{c}
		\begin{lstlisting}[language=SQL]
WITH overwritten AS (SELECT FromRepId, FromCtr FROM pred)
WITH overwrites  AS (SELECT ToRepId, ToCtr FROM pred)
WITH isRoot      AS (SELECT RepId, Ctr FROM set
                    WHERE (RepId, Ctr) NOT IN overwrites)
WITH isLeaf      AS (SELECT RepId, Ctr FROM set
                    WHERE (RepId, Ctr) NOT IN overwritten)
WITH RECURSIVE isCausallyReady AS (
    SELECT * FROM isRoot
    UNION [ALL]
    SELECT pred.ToRepId, pred.ToCtr
    FROM pred, isCausallyReady
    WHERE pred.FromRepId = isCausallyReady.RepId
    AND pred.FromCtr = isCausallyReady.Ctr
)

SELECT set.key, set.value
FROM set, isCausallyReady, isLeaf
WHERE set.RepId = isCausallyReady.RepId
AND set.Ctr = isCausallyReady.Ctr
AND isCausallyReady.RepId = isLeaf.RepId
AND isCausallyReady.Ctr = isLeaf.Ctr\end{lstlisting}
		% \end{tabular}
		\caption{The \ac{MVR} key-value store with causal broadcast in SQL.}\label{code:mvr-crdt-sql}
	\end{subfigure}
\end{figure}

Moreover, the example demonstrates why atomic writes to the underlying database
are important.
Write \(w_2\) updates both the \code{set} and \code{pred} EDBs.
If the query reads state in which only the \code{set} EDB has been updated
but not the \code{pred} EDB,
the query incorrectly deems \setop{4}{r_1}{k_1}{y_3} as a new root and
again returns the invalid result from above.

While the issue of this section can be ignored by assuming a causal broadcast
either on the application or on the database layer,
I think that queries benefit from having the full causal history available.
It provides queries with the ability to detect when operations are concurrent,
and use that information to adjust their conflict handling to perform
custom resolution logic.

\section{Why \emph{\acp{CRDT}} as Queries?}

Next to the higher-level abstraction provided by a query language like Datalog,
there is another important property that applies to \acp{CRDT} in particular.
\acp{CRDT} must satisfy strong eventual consistency to guarantee convergence
in a coodination-free environment.
Proving the correctness of a \ac{CRDT} is a non-trivial
task~\cite{gomes2017verifying, kleppmann2022assessing},
and requires expertise in eventual consistency as well as proving strategies.
This is a prohibitively high entry barrier for application developers
wanting to use \acp{CRDT} in their applications.
With the model of expressing \acp{CRDT} as \emph{deterministic} queries over a
set of gossiped operations among replicas, the convergence property is trivially
satisfied, as I lay out in more detail in \autoref{ch:background}.
I hope that with this approach, the doors to using \acp{CRDT} in applications
are opened wider, as the application developer does not have to worry about
the critical convergence property anymore but is still able to design her own
\ac{CRDT} tailored to her application's needs, which is not possible with
today's \ac{CRDT} libraries.
They provide a fixed set of \ac{CRDT} types with a fixed set of operations
and conflict resolution strategies.
Alas, they are mostly treated as black boxes from the outside.

Similar to how queries made data retrieval and storage more accessible to
application developers, I hope that expressing \acp{CRDT} as queries
makes collaborative software more accessible, too.
Another benefit is that application developers are already used to the concept
of a query, as nearly every application relies on a database in some form.
Furthermore, \ac{CRDT} queries and non-\ac{CRDT} queries can then share the
same interface~\cite{litt2023riffle}, reducing the cognitive load and the
complexity of the application stack.

\section{Query Execution}

The declarative nature of a query language leaves the execution to a query engine.
From the perspective of the application developer this is another advantage:
The exact implementation comes for free and many query engines are heavily
parallelized and optimized for performance. While the latter may be true
for some \ac{CRDT} libraries, they are often not taking advantage of the many
cores of today's computers.
Additionally, the abstraction level of a query language is quite high,
which has the advantage that the query engine can be optimized over time without
introducing breaking changes to the application.

Yet, operation-based \acp{CRDT}, which derive their state from a monotonically
growing set of operations, pose an additional challenge to query engines though.
For each evaluation of the current state of the data structure,
traditional query engines have to re-evaluate the \emph{entire}
set of operations to derive the current state, performing the same work over
and over again.
With tight time budgets in the context of user interfaces, this can lead to
poor user experiences after exceeding a certain threshold of operations.
To address this issue, I focus on exploring \ac{IVM} for this work.

The promise of \ac{IVM} is that it can incrementally maintain a view defined
by a query, while only doing work relative to the size of the query's
inputs' \emph{change since the last evaluation}, as opposed to doing work
relative to the size of the query's \emph{full inputs} (which are monotonically
growing in this case) happening with traditional query engines.
The issue is the stateless nature of non-incremental query engines, or rather
that they ``forget'' any previous work done for evaluations of the query,
and start from scratch every time.
This behavior is not a good fit with most \ac{CRDT} workloads.
They are often used in a near-real-time collaboration settings in which there
are frequent updates both from the local and remote replicas, and the accrued
update sizes since the last evaluation are often minor compared to the total
size of the operation set due to the near-real-time nature of the collaboration.
Only after extended periods of offline time, the update sizes may grow larger.
Therefore, I see \ac{IVM} as a key technique to make \acp{CRDT} as queries
feasible in practice:
If the query engine is only doing work relative to the amount of operations
accrued since the last evaluation instead of relative to the full set of
operations, the query can be evaluated within acceptable time budgets,
irrespective of the size (and correlatedly age) of the operation history.

\section{Contributions}

% Zero: Higher-level abstraction through declarative queries. + Datalog
% First: Convergence for free.
% Second: IVM and implementation for free from the perspective of the app developer.

Having motivated the idea of expressing \acp{CRDT} as deterministic queries to
(1) gain a higher-level abstraction and (2) to trivially satisfy \ac{CRDT}'s
convergence property,
I point out the two main contributions of this work.
First, I present an incremental Datalog query engine based on DBSP~\cite{budiu2022dbsp}.
Next to working incrementally, the engine also handles the conversion from
declarative Datalog to more imperative trees of relational algebra.
Second, I explore expressing \acp{CRDT} as Datalog queries, and evaluate
their performance using the incremental engine.
Finally, I discuss the results and the implications of this approach,
and point out avenues for future work required to render \acp{CRDT} as queries
a feasible alternative to today's \ac{CRDT} libraries.

Ideally, this different approach to \acp{CRDT} would move them closer to the power,
guarantees and flexibility of database systems.
The higher abstraction of a query language better supports the decoupling from
logical and physical data representations than object-oriented APIs.
Moreover, higher abstractions allow for optimizations without breaking changes,
something relational databases have benefitted from over the years and has
arguably contributed towards their widespread adoption.
The data stored in \acp{CRDT} can benefit from the durability guarantees of
database systems, which are often not provided by \ac{CRDT} libraries.
Furthermore, the read finality issue of \acp{CRDT} under non-monotone queries
may be better addressed with the support of a query language~\cite{laddad2022keep}.
From another perspective, applications benefit from extending the idea of functional
reactive programming beyond the \ac{GUI} to the whole application stack,
\emph{including} the database layer.
The application can then be considered a reactive, pure function
of some state captured in a (potentially shared) ground truth~\cite{litt2023riffle}.

On the other hand, database systems can be introduced to coordination-free
environments.
In exchange for some guarantees which cannot be upheld in such environments,
for instance uniqueness constraints,
coordination-free database systems can deliver ultimate availability
to their clients:
As long as the client, which \emph{is} a replica of the distributed database,
is alive, the system is available for reads and writes.
This property can be useful in contexts where a network partition would be
prohibitively expensive to tolerate such as manufacturing processes,
which cannot afford to stop an entire production facility just because
some server is not available.
Although \acp{CRDT} are mostly discussed in the context of collaborative
applications, they are not limited to such contexts.
Companies like Ditto~\cite{ditto} serve use cases powered by \acp{CRDT}
in manufacturing, aviation, gastronomy and defence sectors.

% The goals of this research are twofold:
% First, we deal with the question if Datalog is expressive and ergonomic enough
% to express common CRDTs.
% To answer that question, we implement common CRDT types in Datalog, e.g.,
% a list/sequence CRDT, a map CRDT, as well as a set CRDT,
% and evaluate our experience.
% Queries often involve negation (see Section \ref{sec:motivating-example} and
% \ref{sec:advanced_example}),
% and because core Datalog does not support negation, the question arises if
% the most commonly used Datalog extension for negation,
% stratified negation~\cite{green2013datalog}, is expressive enough.

% Second, we turn to the question of the practical performance of this approach.
% We want to test some CRDT workloads against our CRDT Datalog implementations and
% evaluate the performance.
% To do so, we (1) may have to define common CRDT workloads for benchmarking
% (unfortunately, there is no standard like TPCH established yet\footnote{
% 	There are some benchmarks for text editing:
% 	\url{https://github.com/dmonad/crdt-benchmarks} and
% 	\url{https://github.com/josephg/editing-traces}.
% }), and we (2) either utilize some existing Datalog engines and test them
% with and without incremental view maintenance or may have to deal with
% prototyping our own, as support for incremental view maintenance is variable
% among Datalog engines
% (see \autoref{tab:datalog-engines} for a non-exhaustive overview).

% \begin{table*}[]
% 	\center
% 	\small
% 	\begin{tabular}{@{}lp{3.4cm}lp{1.2cm}p{0.7cm}l@{}}
% 		\toprule
% 		Engine                                                                      & Datalog Variant                          & IVM & Language  & Open Source & Active \\
% 		\midrule
% 		\href{https://github.com/souffle-lang/souffle}{Souffle}                     & With stratified negation                 & No  & C++       & Yes         & Yes    \\
% 		\href{https://github.com/rust-lang/datafrog}{Datafrog}                      & Vanilla                                  & No  & Rust      & Yes         & No     \\
% 		\href{https://github.com/vmware/differential-datalog}{Differential Datalog} & ?                                        & Yes & Rust/Java & Yes         & No     \\
% 		\href{https://github.com/s-arash/ascent/}{Ascent}                           & With stratified negation and aggregation & No  & Rust      & Yes         & Yes    \\
% 		\href{https://github.com/ekzhang/crepe}{Crepe}                              & With stratified negation                 & No  & Rust      & Yes         & No     \\
% 		\href{https://github.com/knowsys/nemo}{Nemo}                                & Based on RDF instead of relations        & ?   & Rust      & Yes         & Yes    \\
% 		\href{https://www.datomic.com}{Datomic}                                     & Custom (and appears to use RDF)          & ?   & Clojure   & No          & Yes    \\
% 		\href{https://github.com/tonsky/datascript}{Datascript}                     & Appears to mimic Datomic                 & ?   & Clojure   & Yes         & Yes    \\
% 		\href{https://github.com/comnik/declarative-dataflow}{Declarative Dataflow} & ?                                        & Yes & Rust      & Yes         & No     \\
% 		\bottomrule
% 	\end{tabular}
% 	\caption{Overview of some Datalog Engines}
% 	\label{tab:datalog-engines}
% \end{table*}
