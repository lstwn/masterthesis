% !TeX root = ../main.tex
% Add the above to each chapter to make compiling the PDF easier in some editors.

\chapter{Background}\label{ch:background}

\section{Datalog}

Datalog~\cite{green2013datalog} is a declarative logic programming language
invented in the 80s which is primarily used for expressing queries to retrieve
data in database systems.
A Datalog program (or query) consists of a set of rules, which are used to
derive new facts from existing ones.

\subsection{Syntax and Semantics}

Rules define \emph{predicates} and are syntactially expressed in the form of
Horn clauses, which are logical implications adhering to this structure:

\begin{equation}
	\underbrace{
	\underbrace{r}_{\text{head}}
	\text{ :- }
	\underbrace{
	\underbrace{a_1}_{\text{atom}},
	\underbrace{a_2}_{\text{atom}},
	\ldots,
	\underbrace{a_n}_{\text{atom}}.
	}_{\text{body}}
	}_{\text{rule}}
\end{equation}

The left-hand side of a rule is called \emph{head} and the right-hand side \emph{body}.
A head consists of an identifier, which also defines the name of the predicate
the rule defines, as well as a comma-separated list of expressions
which may reference variables defined in the rule's body.
A body is a comma-separated sequence of \emph{atoms} followed by a trailing ``.'' (dot).
An atom is either referencing another \emph{predicate} to bring some of its
variables into scope or imposing a boolean condition.
A condition can either restrict a variable's value range (e.g. \(x = 3\))
or specify a relationship with another variable (e.g. \(x = y\)).

A rule (a predicate) is said to be \emph{self-recursive}
if it references itself in its (one of its) body (bodies).
Moreover, it is allowed to have multiple rules sharing the same head, that is,
they have the same identifier and list of expressions.

\ref{code:trans-closure-datalog} illustrates the concepts for the
computation of a graph's transitive closure with Datalog.
It contains three rules which define the two predicates, ``edge'' and ``closure''.
The ``edge'' predicate contains just the expression ``true''.
This denotes an empty body and some Datalog variants allow to omit it,
leaving just the dot on the rule's right-hand side.
The ``closure'' predicate is defined through the latter two rules which share
the same head.
The last rule consists of three atoms: The first one references itself, rendering
the rule self-recursive, the second one references the ``edge'' predicate,
and the last atom defines a condition, restricting the value range of the
``active'' variable.
The first two atoms bring the variables ``from'', ``via'', ``to'', and ``active''
into scope.
The appearance of the same variable ``via'' in both atoms implies that their
values must be equal, i.e.,
\(a_1(x), a_2(x)\) is a shorthand for \(a_1(x_1), a_2(x_2), x_1 = x_2\).
Next to defining the name of the predicate, the head specifies that the ``from''
and ``to'' variables are exposed for the ``closure'' predicate.

\begin{figure}[htpb]
	\centering
	\begin{tabular}{c}
		\begin{lstlisting}[keepspaces]
edge(from, to, active) :- .
closure(from, to)      :- edge(from, to, active), active = true.
closure(from, to)      :- closure(from, via), edge(via, to, active),
                          active = true.\end{lstlisting}
	\end{tabular}
	\caption{An exemplary computation of a graph's transitive closure with Datalog.}\label{code:trans-closure-datalog}
\end{figure}

Semantically, a rule can be read from right to left: The body's atoms are
connected with conjunctions and all variable assignments that satisfy that term
form a new \emph{fact} of the predicate which the rule defines.
Mathematically, this can be read as
``$a_1$ and $a_2$ and \ldots\ and $a_n$ imply $r$'', and expressed as:

\begin{equation}
	\begin{aligned}
		                & r \Leftarrow a_1 \land a_2 \land \ldots \land a_n          \\
		\Leftrightarrow & r \lor \lnot a_1 \lor \lnot a_2 \lor \ldots \lor \lnot a_n
	\end{aligned}
\end{equation}

If there are multiple rules with the same head, their bodies' conjunctions
are connected through a disjunction~\cite{abo2024convergence}.
\emph{Fact rules} are special rules without a body (with $n=0$), e.g.
\( r \text{ :- } .\), and they are assumed to be unconditionally true.
Unlike \emph{derived facts}, \emph{base facts} of fact rules are externally
given and not derived from other rules.
Hence, the predicates defined by fact rules and the predicates defined by regular
rules (those with a body) can be partioned into two disjoint sets,
\emph{\acp{EDBP}} and \emph{\acp{IDBP}}. Only \acp{IDBP} may reference \acp{EDBP}
but not vice-versa.
Facts contain \emph{fields} that can assume basic scalar types such as strings,
number types, or booleans.
Moreover, for a regular rule to be valid the \emph{range-restriction property}
must be satisfied~\cite{green2013datalog}.
It demands that every variable occuring in the head of a rule must also occur
at least once in a predicate atom of its body.

Allowing rules to be self-recursive equips Datalog with the ability to express
repeated computations, therefore requiring a termination condition.
Datalog uses least-fixpoint semantics, under which the computation terminates
if an additional iteration of the repeated computation does not alter the result
anymore \emph{for the first time}.
This has some consequences but before discussing them we are now in position
to illustrate the semantics of the example query in \ref{code:trans-closure-datalog}.

The ``edge'' \ac{EDBP} contains all edges of a graph and these are given
externally, e.g., through an insertion into the database.
The predicate defines the fields ``from'', ``to'', and ``active''.
The former two specify \emph{from which node to which other node} an edge points to.
The latter field can be used to enable or disable an edge but this is just
a contrived example to demonstrate a condition on a variable.
The ``closure'' \ac{IDBP} specifies the actual computation of the graph's
transitive closure.
The first rule of its definition specifies the computation's starting point,
that is, all pairs of nodes which (1) are reachable through a path length of 1
and (2) are marked as ``active''.
The second rule takes all node pairs identified so far, which are reachable
through paths of length \(n\), and discovers new node pairs connected
through a path of length \(n + 1\),
again while restricting the edges to be marked as ``active''.
The computation terminates if a (re)application of the rules on top of the
currently discovered facts does not produce more facts,
upon which the least-fixpoint is attained.

Least-fixpoint computations can only solve problems whose solutions
are monotonically growing, as otherwise the least-fixpoint solution may not
terminate at the optimal solution.
This can be thought of being stuck at a local minimum instead of finding
a global optimum in non-convex optimization.
Another issue, not necessarily tied to least-fixpoint computations but shared
with all termination criteria, is that it is impossible to prevent
non-terminating computations in general.
Consider a small variation on the computation in \ref{code:trans-closure-datalog}.
If every edge carried a non-negative weight and we wanted to keep track of the
cumulated weight for each node pair of the transitive closure,
the computation's termination hinges on the graph being cycle-free.
With cycles, the computation would walk cycles endlessly and constantly discover
higher cumulated weights for node pairs which are part of a cycle.
A discussion of alternatives to fixpoints in the context of SQL
can be found in \cite{hirn2023fix}.

% TODO:
% - mention PR on DBSP (maybe later?)

Traditionally, Datalog has set semantics and does not support multisets.
Moreover, it does not use name-based-indexing but positional-indexing.
As Datalog lacks a formal specification, there are many flavours of it,
and later we define our own dialect of it which supports multisets and
uses name-based-indexing.

Due to the use of relational algebra as an \ac{IR} in \ref{ch:results},
we point out some connections to it as well as to SQL
because the latter is the predominant frontend to relational algebra.
The equivalent of a \emph{relation} in Datalog is a predicate:
They both define a name under which \emph{tuples} (facts in Datalog; rows in SQL)
are made available. SQL calls this construct a \emph{table}.
Yet, relational algebra does not distinguish between base facts and derived facts
the same way Datalog does.
The closest counterpart to derived facts are the tuples of a materialized view
in SQL.
We suspect that this is due to the little emphasis on composition in SQL.

\subsection{Negation Extension}

The Datalog semantics presented so far is not sufficient to express
(among other things) relational algebra's set difference operator
and the advanced example in \ref{fig:mvr-crdt} from \ref{ch:intro}
due to its lack of negation.
However, introducing negation without any restrictions to Datalog causes two issues.
First, every variable occuring in the body of a rule must occur in at least one
positive (non-negated) predicate atom, which is referred to as
\emph{safety condition}~\cite{green2013datalog}.
If it was not satisfied, the result of a query would not just depend on the
actual content of a database, and may be infinite.
Second, if negation is combined with recursion its semantics can become undefined
as \ref{code:negated-datalog-issue} demonstrates.

\begin{figure}[htpb]
	\centering
	\begin{tabular}{c}
		\begin{lstlisting}[keepspaces]
// TODO: Is there a better example? Maybe with some real-world semantics?
r_1 :- not r_2.
r_2 :- not r_1.\end{lstlisting}
	\end{tabular}
	\caption{A negated Datalog program with unclear semantics~\cite{green2013datalog}.}\label{code:negated-datalog-issue}
\end{figure}

The most common resolution to reimpart clear semantics to negative Datalog
is called \emph{stratified negation}.

Introduce semipositive Datalog.

To compute a stratification of a negative Datalog program \(P\),
its \emph{precedence graph} \(G_P\) can be utilized.
It is a directed graph whose nodes are the predicates of \(P\) and there is
a \emph{positive} edge from predicate \(p_i\) to predicate \(p_j\)
if \(p_i\) occurs in the body of a rule defining \(p_j\).
Similarly, there is a \emph{negative} edge from \(p_i\) to \(p_j\)
if \(p_i\) occurs \emph{negatedly} in the body of a rule defining \(p_j\).

\ref{fig:precedence-graph-mvr} shows the precedence graph of \ref{fig:mvr-crdt}.

\section{\acsp{CRDT} and Coordination-Free Environments}

\acp{CRDT}\footnotemark{} try to solve the problems arising from asynchronous
collaboration over message-passing networks on the data structure level.
In these coordination-free environments, in which replicas can be offline
for extended periods of time but are still permitted to write (and read)
to (from) their local state at any time,
the naive delivery of messages allows messages to be arbitrarily delayed,
to be reordered, and the same message may be delivered multiple times.
\acp{CRDT} address these challenges by augmenting messages (hereafter updates
to the \ac{CRDT}) with additional metadata (1) to preserve the user's intention
in the ``best'' possible manner and (2) to ensure the convergence of replicas,
even after temporary divergence.

\footnotetext{
	Although \acp{CRDT} are often referred to as \emph{Conflict-free} Replicated
	Data Types, we use the term \emph{convergent} here because the former term
	may be a bit misleading, as conflicts can still occur, e.g., in case of
	concurrent writes to the same ``location'' of a the data type.
	We think that CRDTs are better characterized by their property that diverging
	replicas eventually \emph{converge} to the same state, given the delivery of
	the same set of updates.
	While that state may comprise conflicts, replicas uniformly agree upon them
	(and their order).
}

Compared to distributed systems that require coordination,
this model is great for two reasons.
First, it offers excellent availability and latency for writes because
they do not have to coordinate over the network with other replicas.
Second, reads also offer excellent latency but may be stale insofar that they
do not always reflect the full global state.
As a consequence of allowing offline writes,
application specific invariants (beyond the ones guaranteed by the concrete \ac{CRDT})
may be violated on the aggregate level after convergence,
even though each write individually respected the invariants (based on their
respective replica's state at write creation time).
However, this issue sometimes even affects the ``guarantees'' provided by a \ac{CRDT}:
Their correct definition is complex and error-prone,
let alone proving their convergence, even for people familiar with distributed
systems~\cite{kleppmann2022assessing, gomes2017verifying}.
Hence, we want to explore defining CRDTs through deterministic queries over
relational data.
This ensures that the convergence property is always satisfied by construction,
thereby allowing application developers with little background in eventual
consistency to formulate their own, application-specific \acp{CRDT} without
having to worry about convergence.

We provide a system overview of the approach in \autoref{fig:system-arch}.
The \emph{database layer} incrementally updates the views \deltaO{},
as defined by the CRDT Datalog queries formulated on the \emph{application layer},
in response to updates from both the local \deltaI{local} and remote replicas
\deltaI{remote}.
In this model the application layer is responsible for forwarding updates
to the database layer but forwarding could equally well happen exclusively
on the latter.
The database layer must be capable of atomically updating all relations
of a write to safeguard the CRDT queries against reading inconsistent
state, i.e., queries must read from a consistent snapshot of the database.
The next section demonstrates the importance of atomic writes.

\autoref{fig:system-arch} illustrates a peer to peer architecture
but, due to the lack of hierarchy, different network topologies are also possible,
e.g., a star network topology with a central server for more efficient update
gossiping.
Yet, the issue of update propagation (and update integrity) is not the focus
of this research and various approaches exist in the literature~\cite{
	auvolat2019merkle, sanjuan2020merkle, kleppmann2024bluesky,
	kleppmann2022making}.
This also excludes the question of how to efficiently send just the minimum
set of missing updates to each replica.
We only make the basic assumption of some network topology and protocol
which ensures that each update will eventually be delivered to all replicas.
Replicas are assumed to be non-byzantine.

\newcommand{\replica}{
	\begin{tikzpicture}
		\scriptsize
		\tikzset{
		layer/.style={
				rectangle, draw, minimum height=1.5cm, minimum width=2.3cm,
				rounded corners=1mm,
				font=\footnotesize, align=center,
			},
		rel/.style={
		->, >={Stealth[round]}, },
		}
		\node[layer] (app) {\textbf{Application}\\ \textbf{Layer}};
		\node[layer,right=of app] (db) {
			\textbf{Database Layer}\\
			Maintains Datalog\\
			Queries \deltaI{} \(\to\) \deltaO{}
		};
		\node[layer,above=2.0cm of app] (user) {\textbf{\ac{GUI}}};

		\draw[rel] (user) to[bend left] node[midway,near start,auto=left,yshift=-7pt]{emits new \deltaI{local}} (app);
		\draw[rel] (app) to[bend left] node[auto,sloped]{integrates \deltaO{}} (user);

		\draw[rel] (app)
		to[bend right=45, auto=right]
		node[midway]{forwards \deltaI{local} and \deltaI{remote}}
		(db);
		\draw[rel] (db)
		to[bend right=45, auto=right]
		node[midway, near start]{derives \deltaO{}}
		(app);
	\end{tikzpicture}
}

\begin{figure}
	\centering

	\begin{tikzpicture}
		\tikzset{
		replica/.style={rectangle, draw, rounded corners=1mm,fill=white},
		local_replica/.style={replica},
		remote_replica/.style={replica,scale=0.66},
		label/.style={font=\bfseries},
		rel/.style={
		->, >={Stealth[round]},bend left,},
		}

		\node[local_replica] (local) {\replica{}};
		\node[label] (local-label) [above=0pt of local] {Local Replica};

		\node[remote_replica] (remote3) [right=of local,yshift=+15pt,xshift=+15pt] {\replica{}};
		\node[remote_replica] (remote2) [right=of local,yshift=+10pt,xshift=+10pt] {\replica{}};
		\node[remote_replica] (remote1) [right=of local,yshift=+5pt,xshift=+5pt] {\replica{}};
		\node[remote_replica] (remote0) [right=of local] {\replica{}};
		\node[label] (remote-label) [above=0pt of remote3] {Remote Replicas};

		\begin{scope}[on background layer]
			\draw[rel] (local.north east) to[] (remote0.north west);
			\draw[rel] (local.north east) to[] (remote1.north west);
			\draw[rel] (local.north east) to[] (remote2.north west);
			\draw[rel] (local.north east) to[] node[midway, above, yshift=5pt]{distributes \deltaI{local}} (remote3.north west);

			\draw[rel] (remote0.south west) to[] (local.south east);
			\draw[rel] (remote1.south west) to[] (local.south east);
			\draw[rel] (remote2.south west) to[] (local.south east);
			\draw[rel] (remote3.south west) to[] node[midway, below, xshift=40pt]{receives \deltaI{remote}} (local.south east);
		\end{scope}

	\end{tikzpicture}

	\caption{
		Overview of the system architecture from the perspective of a local
		replica.
	}\label{fig:system-arch}
\end{figure}


% \subsection{Connecting the Dots: CRDTs and Datalog}\label{sec:crdts-datalog}

The literature defines two classes of CRDTs: state-based and operation-based
which both have different requirements for correctness.
Let \( S \) be the set of all possible states of a CRDT.
State-based CRDTs require a merge function \( \sqcup: S \times S \to S \)
which must be commutative, associative, and idempotent.
Operation-based CRDTs require that the operation functions \( op_i: S \to S \)
are commutative and applied exactly once.
Both models must adhere to these properties under the strong eventual consistency
model which demands three properties~\cite{shapiro2011comprehensive}:

\begin{enumerate}
	\item \textbf{Eventual Delivery}: All updates are eventually delivered to
	      all replicas.
	\item \textbf{Termination}: All method executions terminate.
	\item \textbf{Convergence}: All replicas that have delivered the same set of
	      updates are in an equivalent state.
\end{enumerate}

Verifying the correctness for a CRDT is a complex,
error-prone task~\cite{gomes2017verifying, kleppmann2022assessing},
and currently has to be done for each CRDT individually.
If, however, the set of operations on a CRDT \emph{is} the state,
the merge function can be defined as the set union,
for which the properties of commutativity, associativity, and idempotency hold.
The critical convergence property demanded by the strong eventual consistency
model is then also trivially satisfied because the state is by definition
the set of all (delivered) operations.
Moreover, applying any \emph{pure}\footnote{
	That is, the function is deterministic and side-effect-free.
}
function \( f: S \to T \) on the state does not impede the convergence property,
as the function is applied on all replicas in the same deterministic way.
\( T \) is an arbitrary set of all possible derived states \( f \) can map to.
This approach to CRDTs is known in the literature as
\emph{pure operation-based replicated data types}~\cite{baquero2017pure, stewen2024undo},
and is used in practice in the Automerge CRDT~\cite{automerge}.

The pure function \( f \) can be expressed in a Datalog query.
As Datalog evaluation is deterministic, the convergence property is satisfied.
This gives application developers the power to define their own, custom CRDTs
without worrying about the complex aspect of convergence.
Furthermore, the query engine running the Datalog program is responsible
for finding an efficient algorithm to execute the query and application developers
can focus on the CRDT at a declarative level and get the implementation ``for free''.

A common pattern of collaboration is near-real-time collaboration on a shared
document and CRDTs can be used to power this form of collaboration.
In this scenario, updates are usually frequent but small and therefore it
may make sense to reuse the result of a previous query along with the input
delta in hope of saving computation time over recomputing the query from scratch
based on the whole input with every update.
Ideally, the computation time is only proportional to the size of the input
change but not to the whole input size anymore.
As older, causally overwritten updates to a CRDT are often not contributing
to the current state of the CRDT anymore, incremental view
maintenance (IVM)~\cite{mcsherry2013differential, budiu2022dbsp, budiu2024dbsp}
can turn out to be an essential aspect to make CRDTs expressed as
queries over relations feasible.

\section{Incremental View Maintenance}

Introduce DBSP's concepts: Circuits, operators, handles.
