% !TeX root = ../main.tex
% Add the above to each chapter to make compiling the PDF easier in some editors.

\chapter{Evaluation}\label{ch:evaluation}

This chapter assesses the suitability of my Datalog dialect from
\ref{sec:datalog-frontend} to express \acp{CRDT}-as-queries
in \ref{sec:crdts-as-queries}, by implementing two classes of \acp{CRDT} in it.
To evaluate the approach's viability in practice,
\ref{sec:benchmarks} presents benchmarks of the implemented \acp{CRDT}
which are executed on my query engine from \ref{ch:implementation}.

\section{\acp{CRDT} as Queries}\label{sec:crdts-as-queries}

\ref{sec:key-value-stores-datalog-dialect} shows the familiar \ac{MVR} key-value
stores from \ref{ch:intro} in my Datalog dialect.
\ref{sec:list-crdt-datalog-dialect} implements a list \ac{CRDT} in my Datalog
dialect, which is adapted from~\cite{kleppmann2018data}.

\subsection{Key-Value Stores}\label{sec:key-value-stores-datalog-dialect}

\ref{code:mvr-store-datalog-dialect} shows the equivalent of the \ac{MVR}
key-value store from \ref{code:mvr-store-datalog} in my Datalog dialect.
Examples in this chapter explicitly state the \acp{EDBP} to communicate the
predicates' schema to the query engine.

\begin{figure}[htpb]
	\begin{lstlisting}[keepspaces]
// EDBPs:
pred(FromRepId, FromCtr, ToRepId, ToCtr) :- .
set(RepId, Ctr, Key, Value)              :- .

// IDBPs:
distinct overwritten(RepId, Ctr)
                    :- pred(RepId = FromRepId, Ctr = FromCtr).
mvrStore(Key, Value)
                    :- set(RepId, Ctr, Key, Value),
                       not overwritten(RepId, Ctr).\end{lstlisting}
	\caption{The \ac{MVR} key-value store in my Datalog dialect.}\label{code:mvr-store-datalog-dialect}
\end{figure}

\ref{code:mvr-crdt-datalog-dialect} provides a definition of the \ac{MVR}
key-value store \ac{CRDT} from \ref{code:mvr-store-datalog} in my Datalog dialect.
The query differentiates itself from the previous example by including the
causal broadcast mechanism.

\begin{figure}[htpb]
	\begin{lstlisting}[keepspaces]
// EDBPs are omitted because they are shared with the previous example. IDBPs:
distinct overwritten(RepId, Ctr)
                    :- pred(RepId = FromRepId, Ctr = FromCtr).
distinct overwrites(RepId, Ctr)
                    :- pred(RepId = ToRepId, Ctr = ToCtr).
isRoot(RepId, Ctr)  :- set(RepId, Ctr, _Key, _Value),
                       not overwrites(RepId, Ctr).
isLeaf(RepId, Ctr)  :- set(RepId, Ctr, _Key, _Value),
                       not overwritten(RepId, Ctr).
isCausallyReady(RepId, Ctr)
                    :- isRoot(RepId, Ctr).
isCausallyReady(RepId, Ctr)
                    :- isCausallyReady(FromRepId = RepId, FromCtr = Ctr),
                       pred(FromRepId, FromCtr, RepId = ToRepId, Ctr = ToCtr).
mvrStore(Key, Value)
                    :- isLeaf(RepId, Ctr),
                       isCausallyReady(RepId, Ctr),
                       set(RepId, Ctr, Key, Value).\end{lstlisting}
	\caption{The \ac{MVR} key-value \ac{CRDT} store in my Datalog dialect.}\label{code:mvr-crdt-datalog-dialect}
\end{figure}

Both examples are similar to their counterparts from \ref{ch:intro}
expressed in ``conventional'' Datalog.
Key differences are the inclusion of \acp{EDBP}, the use of name-based indexing,
and the explicit use of the ``distinct'' operator to ensure that a predicate
has set (instead of multiset) semantics.

\subsection{List \ac{CRDT}}\label{sec:list-crdt-datalog-dialect}

List \acp{CRDT} are more complex than key-value stores because they have to
converge to the \emph{same order} of elements across replicas under concurrent
updates.
In the more specific context of text editing, Treedoc~\cite{treedoc},
Logoot~\cite{logoot}, RGA~\cite{rga}, and Fugue~\cite{fugue} have been proposed,
among others.
The list \ac{CRDT} presented here resembles RGA's design,
which is based on the idea that insertions do not specify an unstable list index
for their position but instead each list element is assigned a stable identifier,
and each insertion references the identifier of the element after which it wants
to be inserted.
This implies that deletions cannot fully remove an element, but have to leave
\emph{tombstones} behind, to avoid dangling references in case of an
insertion after element \(x\) and a concurrent deletion of \(x\).
In case of concurrent insertions after the same element \(x\),
the inserted elements are ordered according to the total order induced
by the stable identifiers.
The stable identifiers can again be replica id and counter pairs,
and they provide a total order by first comparing the counters and,
in case of ties, by breaking them through comparing the replica ids.
The ordering brings up the issue of \emph{non-interleaving} the elements from
the same replica with elements from other replicas under concurrent insertions,
a desirable property for maintaining the users' intents in the best
``possible'' manner, and is discussed in~\cite{fugue}.

Logically, insertions in RGA can be represented in a tree structure,
where each node's \emph{parent} is the element after which it is supposed to be
inserted (but may not in the final list order due to concurrent insertions).
The root element is some sentinel element that is always present and does not
contribute to the list's content.
A node's \emph{children} are \emph{descendingly} ordered by their identifiers.
The number of children corresponds to the number of concurrent insertions
after the parent element, i.e., siblings are a result of concurrency.
The final list order is given by a depth-first, pre-order traversal of the tree.
At its core, the Datalog query has to implement this depth-first, pre-order
traversal of the tree.

\begin{figure}[tpb]
	\centering
	\newcommand{\nodelabel}[2]{\((#1, #2)\)}
	\begin{tikzpicture}
		\small

		\tikzset{
			node distance=60pt and 85pt,
			on grid,
			node/.style={
					rectangle,draw,minimum width=0.5cm, minimum height=0cm,
					align=center,fill=white
				},
			parent/.style={
					edge,
					<-,
				},
			firstChild/.style={
					edge,
					TUMAccentGreen,
					densely dotted,
				},
			nextSibling/.style={
					edge,
					TUMAccentOrange,
					dashed,
				},
			nextSiblingAnc/.style={
					edge,
					TUMAccentBlue,
					dashdotted,
				},
			edgelabel/.style={midway,sloped,font=\footnotesize},
		}

		\node[node,] (sentinel) {\nodelabel{0}{0}};

		\node[node,below right=of sentinel] (11) {\nodelabel{1}{1}};
		\node[node,below=of 11] (22) {\nodelabel{2}{2}};

		\node[node,below left=of sentinel] (21) {\nodelabel{2}{1}};
		\node[node,below left=of 21] (23) {\nodelabel{2}{3}};
		\node[node,below=of 21] (13) {\nodelabel{1}{3}};
		\node[node,below right=of 21] (32) {\nodelabel{3}{2}};

		\def\offset{4pt}
		\begin{scope}[on background layer]
			\draw[parent] (sentinel) to[] node[edgelabel,above]{H} (21);
			\draw[parent] (sentinel) to[] node[edgelabel,above]{O} (11);
			\draw[parent] (21) to[] node[edgelabel,above]{E} (23);
			\draw[parent] (21) to[] node[edgelabel,above]{L} (13);
			\draw[parent] (21) to[] node[edgelabel,above]{L} (32);
			\draw[parent] (11) to[] node[edgelabel,above]{!} (22);

			\draw[firstChild] (sentinel) to[bend right] node[edgelabel,above]{firstChild} (21);
			\draw[firstChild] (21) to[bend right] node[edgelabel,above]{firstChild} (23);
			\draw[firstChild] (11.east) to[bend left] node[edgelabel,above]{firstChild} (22.east);

			\draw[nextSibling] (23) to[] node[edgelabel,above]{nextSibling} (13);
			\draw[nextSibling] (13) to[] node[edgelabel,above]{nextSibling} (32);
			\draw[nextSibling] ([yshift=\offset]21.east) to[] node[edgelabel,above]{nextSibling} ([yshift=\offset]11.west);

			\draw[nextSiblingAnc] (23) to[bend right] node[edgelabel,below]{nextSiblingAnc} (13);
			\draw[nextSiblingAnc] (13) to[bend right] node[edgelabel,below]{nextSiblingAnc} (32);
			\draw[nextSiblingAnc] (21) to[] node[edgelabel,below]{nextSiblingAnc} (11);
			\draw[nextSiblingAnc] (32) to[] node[edgelabel,below]{nextSiblingAnc} (11);
		\end{scope}
	\end{tikzpicture}
	\caption{
		An example tree spanned by insertions into the list \ac{CRDT}.
	}\label{fig:list-crdt}
\end{figure}


\ref{code:list-crdt-datalog-dialect} shows the definition of an RGA-like
list \ac{CRDT} in my Datalog dialect.
To explain the query, I use \ref{fig:list-crdt} as an example in which
three replicas, with id 1, 2, and 3, insert elements into the list.
\ref{fig:list-crdt} visualizes the ``insert(RepId, Ctr, ParentRepId, ParentCtr)''
\ac{EDBP} with entries:

\[\{ (2,1,0,0), (2,3,2,1), (1,3,2,1), (3,2,2,1), (1,1,0,0), (2,2,1,1) \}\]

Every node (except for the sentinel root) depicts a (\textit{RepId}, \textit{Ctr})
pair and its black edge points to its parent defined by
a (\textit{ParentRepId}, \textit{ParentCtr}) pair.
Siblings are ordered \emph{descendingly} according to their stable identifiers.
The result of the depth-first, pre-order traversal is:

\[[ (0,0), (2,1), (2,3), (1,3), (3,2), (1,1), (2,2) ]\]

To obtain this result with Datalog, the ``nextElem'' \ac{IDBP} is used in
\ref{code:list-crdt-datalog-dialect}.
Its definition is divided into three parts.
First, the ``firstChild'' \ac{IDBP} (\ref{code:list-crdt-datalog-dialect-part1})
finds the first child of each parent
(the dotted, green edges in \ref{fig:list-crdt}).
To do so, it relies on the ``laterChild'' \ac{IDBP} to find all parents with
their children such that the children are not their first but a later child.
Second, the ``nextSibling'' \ac{IDBP} (\ref{code:list-crdt-datalog-dialect-part2})
provides the next sibling of each child
(the dashed, orange edges in \ref{fig:list-crdt}).
Its definition makes use of the ``laterSibling'' and ``laterIndirectSibling''
\acp{IDBP}. The former finds all later siblings of a child, and the latter
finds all later siblings of a child that are not direct siblings but have
at least one sibling in between them.
Then, the next sibling of a child is given by filtering out all later
\emph{indirect} siblings from the later siblings.
Third, the ``nextSiblingAnc'' \ac{IDBP} (\ref{code:list-crdt-datalog-dialect-part3})
gives the next sibling of each child, or, recursively,
the next sibling of the parent if a child has no next sibling
(the dashdotted, blue edges in \ref{fig:list-crdt}).
Finally, the ``nextElem'' \ac{IDBP} defines the depth-first, pre-order traversal
by either yielding a link from the parent to its first child (as ``previous''
and ``next'', respectively) or a link defined by the ``nextSiblingAnc'' \ac{IDBP}
if the ``previous'' node has no child.
This outputs the above result as a linked list of nodes where each node is
defined by a (\textit{RepId}, \textit{Ctr}) pair:

\[[ (0,0,2,1), (2,1,2,3), (2,3,1,3), (1,3,3,2), (3,2,1,1), (1,1,2,2) ]\]

TODO:
How to assign values to the list?
How to handle deletions?
How to output the order with the values?

\begin{figure}[htpb]
	\centering

	\begin{subfigure}[b]{\textwidth}
		\begin{lstlisting}[keepspaces,escapechar=!]
// Black edges in !\ref{fig:list-crdt}!.
insert(RepId, Ctr, ParentRepId, ParentCtr) :- .

laterChild(ParentRepId, ParentCtr, ChildRepId, ChildCtr) :-
    insert(SiblingRepId = RepId, SiblingCtr = Ctr, ParentRepId, ParentCtr),
    insert(ChildRepId = RepId, ChildCtr = Ctr, ParentRepId, ParentCtr),
    (SiblingCtr > ChildCtr; (SiblingCtr == ChildCtr, SiblingRepId > ChildRepId)).

// Dotted, green edges in !\ref{fig:list-crdt}!.
firstChild(ParentRepId, ParentCtr, ChildRepId, ChildCtr) :-
    insert(ChildRepId = RepId, ChildCtr = Ctr, ParentRepId, ParentCtr),
    not laterChild(ParentRepId, ParentCtr, ChildRepId, ChildCtr).\end{lstlisting}
		\caption{Part 1: Definition of ``insert'' \ac{EDBP} and ``firstChild'' \ac{IDBP}.}\label{code:list-crdt-datalog-dialect-part1}
	\end{subfigure}

	\vspace{1em}

	\begin{subfigure}[b]{\textwidth}
		\begin{lstlisting}[keepspaces,escapechar=!]
sibling(Child1RepId, Child1Ctr, Child2RepId, Child2Ctr) :-
    insert(Child1RepId = RepId, Child1Ctr = Ctr, ParentRepId, ParentCtr),
    insert(Child2RepId = RepId, Child2Ctr = Ctr, ParentRepId, ParentCtr).

laterSibling(Child1RepId, Child1Ctr, Child2RepId, Child2Ctr) :-
    sibling(Child1RepId, Child1Ctr, Child2RepId, Child2Ctr),
    (Child1Ctr > Child2Ctr; (Child1Ctr == Child2Ctr, Child1RepId > Child2RepId)).

laterIndirectSibling(Child1RepId, Child1Ctr, Child3RepId, Child3Ctr) :-
    sibling(Child1RepId, Child1Ctr, Child2RepId, Child2Ctr),
    sibling(Child1RepId, Child1Ctr,
        Child3RepId = Child2RepId, Child3Ctr = Child2Ctr),
    (Child1Ctr > Child2Ctr; (Child1Ctr == Child2Ctr, Child1RepId > Child2RepId)),
    (Child2Ctr > Child3Ctr; (Child2Ctr == Child3Ctr, Child2RepId > Child3RepId)).

// Dashed, orange edges in !\ref{fig:list-crdt}!.
nextSibling(Child1RepId, Child1Ctr, Child2RepId, Child2Ctr) :-
    laterSibling(Child1RepId, Child1Ctr, Child2RepId, Child2Ctr),
    not laterIndirectSibling(Child1RepId, Child1Ctr,
        Child2RepId = Child3RepId, Child2Ctr = Child3Ctr).\end{lstlisting}
		\caption{Part 2: Definition of ``nextSibling'' \ac{IDBP}.}\label{code:list-crdt-datalog-dialect-part2}
	\end{subfigure}
\end{figure}

\begin{figure}[htpb]\ContinuedFloat
	\centering

	\begin{subfigure}[b]{\textwidth}
		\begin{lstlisting}[keepspaces,escapechar=!]
distinct hasNextSibling(ChildRepId, ChildCtr) :-
    nextSibling(ChildRepId = Child1RepId, ChildCtr = Child1Ctr).

// Dashdotted, blue edges in !\ref{fig:list-crdt}!.
nextSiblingAnc(ChildRepId, ChildCtr, AncRepId, AncCtr) :-
    nextSibling(ChildRepId = Child1RepId, ChildCtr = Child1Ctr,
        AncRepId = Child2RepId, AncCtr = Child2Ctr).
nextSiblingAnc(ChildRepId, ChildCtr, AncRepId, AncCtr) :-
    insert(ChildRepId = RepId, ChildCtr = Ctr, ParentRepId, ParentCtr),
    not hasNextSibling(ChildRepId, ChildCtr),
    nextSiblingAnc(ParentRepId = ChildRepId, ParentCtr = ChildCtr,
        AncRepId, AncCtr).

distinct hasChild(ParentRepId, ParentCtr) :-
    insert(ParentRepId, ParentCtr).

nextElem(PrevRepId, PrevCtr, NextRepId, NextCtr) :-
    firstChild(PrevRepId = ParentRepId, PrevCtr = ParentCtr,
        NextRepId = ChildRepId, NextCtr = ChildCtr).
nextElem(PrevRepId, PrevCtr, NextRepId, NextCtr) :-
    not hasChild(PrevRepId = ParentRepId, PrevCtr = ParentCtr),
    nextSiblingAnc(PrevRepId = ChildRepId, PrevCtr = ChildCtr,
        NextRepId = AncRepId, NextCtr = AncCtr).\end{lstlisting}
		\caption{Part 3: Definition of ``nextSiblingAnc'' and ``nextElem'' \acp{IDBP}.}\label{code:list-crdt-datalog-dialect-part3}
	\end{subfigure}
	\caption{A list \ac{CRDT} in my Datalog dialect, adapted from~\cite{kleppmann2018data}.}\label{code:list-crdt-datalog-dialect}
\end{figure}

\section{Performance Evaluation}\label{sec:benchmarks}

\ref{sec:near-real-time-benchmark} and \ref{sec:hydration-benchmark}
provide benchmarks of the three \acp{CRDT} from the previous section.
Benchmarking happens in two different settings which are distinguished by
the operators of the query plan being either ``warm'' or ``cold'':

\begin{itemize}
	\item \textbf{Near-real time setting} (\ref{sec:near-real-time-benchmark}).
	      In this setting, the operators are assumed to be ``warm'' and only
	      a small number of updates are new to them.
	      This benchmark measures the common operation mode of \acp{CRDT} and
	      is therefore the most relevant for practical applications. It asks
	      how much time it takes to process a small number of updates on top of
	      an already ``hydrated'' query plan.
	\item \textbf{Hydration setting} (\ref{sec:hydration-benchmark}).
	      As outlined in \ref{sec:ivm}, all bilinear operators are stateful.
	      Hence, all stateful operators of the query plan have to restore
	      their state before processing any new updates, e.g., in case of ``cold''
	      application restarts. This benchmark measures the time it takes
	      until the application can process \emph{new} updates again on top of
	      an existing history of operations.
\end{itemize}

\subsection{Near-Real-Time Setting}\label{sec:near-real-time-benchmark}

\begin{figure}[htpb]
	\centering
    \begin{tikzpicture}
    \begin{axis}[
        axis on top, ymin=0,
    	legend pos=north east,
    	legend style={
    	    anchor=north west,xshift=15pt,
        },
        grid=major,
    	thick,
    	xlabel={Delta Diameter},
    	ylabel={Median Runtime (ms)},
    	sharp plot,
    ]
    \foreach \i in {1000, 2000, 3000, 4000, 5000} {
        \addplot+[mark=*] table [
            x=delta,
            y=median_point_estimate,
            col sep=semicolon,
        ] {data/bench_nrt_nocb_b\i.csv};
        \addlegendentryexpanded{\ref{code:mvr-store-datalog-dialect}, Base Diameter \i}
    }

    \foreach \i in {1000, 2000, 3000, 4000, 5000} {
        \addplot+[mark=+] table [
            x=delta,
            y=median_point_estimate,
            col sep=semicolon,
        ] {data/bench_nrt_cb_b\i.csv};
        \addlegendentryexpanded{\ref{code:mvr-crdt-datalog-dialect}, Base Diameter \i}
    }
    \end{axis}
    \end{tikzpicture}
    \caption{
        Near-real-time setting benchmark for the key-value stores.
    }\label{fig:near-real-time-kv-store}
\end{figure}


\ref{fig:near-real-time-kv-store} shows benchmarks of the \ac{MVR} key-value
store (\ref{code:mvr-store-datalog-dialect}) and the \ac{MVR} key-value
\ac{CRDT} store (\ref{code:mvr-crdt-datalog-dialect}) in the near-real-time
setting.
Their distinguishing factor is the inclusion of the causal broadcast mechanism
and this can be seen in the performance difference between the two.
The benchmark is set up as follows:
The \emph{base diameter} \(\in \{1000, 2000, 3000, 4000, 5000\}\) is the diameter
of the causal history if viewed as a directed graph,
i.e., the longest path from a root to a leaf.
The \emph{delta diameter} \(\in \{20, 40, 60, 80, 100\}\) is the diameter of the
updates, again viewed as a directed graph.
Hence, an update with a delta diameter of, e.g., 20, contains 20 new operations that
causally depend on each other and are applied on top of the causal history
with the given base diameter.
The benchmark measures the time it takes to only apply the operations from the
update.

\subsection{Hydration Setting}\label{sec:hydration-benchmark}

\begin{figure}[htpb]
	\centering
    \begin{tikzpicture}
    \begin{axis}[
        axis on top, ymin=0,
    	legend pos=north east,
    	legend style={
    	    anchor=north west,xshift=15pt,
        },
        grid=major,
    	thick,
    	xlabel={Base Diameter},
    	ylabel={Median Runtime (ms)},
    	sharp plot,
    ]
    \addplot+[mark=*] table [
        x=diameter,
        y=median_point_estimate,
        col sep=semicolon,
    ] {data/bench_hdr_nocb.csv};
    \addlegendentryexpanded{\ref{code:mvr-store-datalog-dialect}}

    \addplot+[mark=+] table [
        x=diameter,
        y=median_point_estimate,
        col sep=semicolon,
    ] {data/bench_hdr_cb.csv};
    \addlegendentryexpanded{\ref{code:mvr-crdt-datalog-dialect}}
    \end{axis}
    \end{tikzpicture}
    \caption{
        Hydration setting benchmark for the key-value stores.
    }\label{fig:hydration-kv-store}
\end{figure}


TODO:
Show that the causal broadcast does not benefit from parallelization
as it is inherently sequential by benchmarking the non-causal CRDT
as reference.
Lots of dependent joins in the fixed point iteration. How to optimize?

\section{Related Work}\label{sec:related-work}

The idea of using restricted languages to express \acp{CRDT} to guarantee
their convergence has been explored in the past.
VeriFx~\cite{verifx}, Propel~\cite{propel}, and LoRe~\cite{lore} introduce
custom \acp{DSL} to define \acp{CRDT}.
From \ac{CRDT} definitions in the \ac{DSL}, they derive both an implementation
of the \ac{CRDT} and its proof of convergence.
The verification is handed off to a \acs{SAT} solver, which is used to
prove that the \ac{CRDT} converges under concurrent updates.
LoRe differs from VeriFx insofar that it can also express invariants that
require coordination between replicas and inject such coordination logic
automatically, freeing the application developer from writing synchronization
code.
Unlike \acp{DSL}, Datalog has the advantage that is a more widely used language
and does not require verification time because convergence is guaranteed by
construction.

There exist several libraries that implement \acp{CRDT} in imperative programming
languages, for instance, Automerge~\cite{automerge} (Rust),
Yjs~\cite{yjs} (JavaScript), Collabs~\cite{collabs} (TypeScript),
and Loro~\cite{loro} (Rust).
They come with no formal proof of convergence but are crafted by experienced
developers familiar with eventual consistency and are widely used in practice.
Yet, changing them to support custom data types is a non-trivial task and
requires a deep understanding of the library's internals as well as theoretical
foundations of \acp{CRDT}, rendering them less flexible than language based
approaches.

Other \ac{CRDT} frameworks are based on event logs, deriving state from the log,
and replaying events to handle concurrent updates.
The idea has been formalized in \cite{egwalker}.
An example for this approach is LiveStore~\cite{livestore}, which allows
developers to specify events and how to derive state from them.
Yet, the convergence is not guaranteed by construction and this burden is
left to the application developer.

In general, there are two predominant approaches to query execution,
\emph{interpreted} and \emph{compiled} query execution.
The former executes a query by interpreting its query plan at run-time,
usually on the granularity level of an operator of the query plan.
Compiled query execution has been pioneered by the HyPer main memory database
system~\cite{neumann2011efficiently}.
It compiles a query plan into an executable tailored to the specific query,
can therefore take advantage of, e.g., combining multiple non-blocking operators
into a single loop, and avoids interpretation overhead.
While compiled query execution sounds more promising in terms of performance,
it is more complex to implement, debug, and, surprisingly, its performance
is not necessarily better~\cite{kersten2018everything}.
Interestingly, Feldera, the company behind the commercial offering around DBSP,
includes a SQL-to-DBSP compiler which emits a Rust executable to execute a
specific query~\cite{feldera}.
The now-unmaintained Differential Datalog project~\cite{ddlog},
which relies on differential dataflow, also compiles a query plan into a
Rust executable.
The strong preference for compiled query execution is unclear to me and,
as stated in \ref{sec:future-work}, I am interested in seeing their differences
being analyzed more closely, especially in the context of \emph{incremental}
query engines.
