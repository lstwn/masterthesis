\begin{figure}[tpb]
	\centering

	\begin{tabular}{c}
		\begin{lstlisting}[keepspaces,escapechar=!]
edges   <- LiteralExpr(Relation { schema: ["from", "to", "weight"] })
base    <- ProjectionExpr {
             relation: VarExpr("edges"),
             attributes: [
               ("from", !\tikzmark{qtimes1}!VarExpr("from")!\tikzmark{qtimee1}!),
               ("to", !\tikzmark{qtimes2}!VarExpr("to")!\tikzmark{qtimee2}!),
               ("cweight", !\tikzmark{qtimes3}!VarExpr("weight")!\tikzmark{qtimee3}!),
               ("hopcnt", !\tikzmark{qtimes4}!LiteralExpr(UInt(1))!\tikzmark{qtimee4}!),
             ]
           }
closure <- FixedPointIterExpr {
             imports: ["edges"],
             accumulator: ("acc", VarExpr("base")),
             step: BlockStmt {
               stmts: [
                 ExprStmt {
                   expr: EquiJoinExpr {
                     left: AliasExpr { relation: VarExpr("acc"), alias: "cur" },
                     right: AliasExpr { relation: VarExpr("edges"), alias: "next" },
                     on: [(!\tikzmark{qtimes5}!VarExpr("cur.to")!\tikzmark{qtimee5}!, !\tikzmark{qtimes6}!VarExpr("next.from")!\tikzmark{qtimee6}!)],
                     attributes: [
                       ("from", !\tikzmark{qtimes7}!VarExpr("cur.from")!\tikzmark{qtimee7}!),
                       ("to", !\tikzmark{qtimes8}!VarExpr("next.to")!\tikzmark{qtimee8}!),
                       (
                         "cweight",
                         !\tikzmark{qtime91}!BinaryExpr
                           op: Operator::Add,
                           left: VarExpr("cur.cweight"),
                           right: VarExpr("next.weight")!\tikzmark{qtime92}!
                         }!\tikzmark{qtime93}!
                       ),
                       (
                         "hopcnt",
                         !\tikzmark{qtime101}!BinaryExpr {
                           op: Operator::Add,
                           left: VarExpr("cur.hopcnt"),!\tikzmark{qtime102}!
                           right: LiteralExpr(UInt(1))
                         }!\tikzmark{qtime103}!
                       )
                     ]
           }}]}}\end{lstlisting}
	\end{tabular}

	\begin{tikzpicture}[remember picture,overlay]
		\tikzset{
			qtime/.style={
					rectangle,draw,
					TUMAccentBlue,
					rounded corners=2pt,
					line width=0.8pt,
					font=\footnotesize,align=center,
					minimum height=2.2ex,
					yshift=0.55ex,
					xshift=-0.1ex,
					inner sep=0.5pt,
				},
		}
		\begin{scope}[on background layer]
			\foreach \i in {1,...,8}
				{\node[qtime,fit=(pic cs:qtimes\i) (pic cs:qtimee\i)] () {};}

			\foreach \i in {9,10}
				{\node[qtime,minimum height=6*2.0ex,
						fit=(pic cs:qtime\i1) (pic cs:qtime\i2) (pic cs:qtime\i3)
					] (sadf) {};}
		\end{scope}
	\end{tikzpicture}

	\caption{
	    The computation of the transitive closure from \ref{code:trans-closure-datalog}
		translated into the \ac{IR}.
	}\label{code:trans-closure-ir}
\end{figure}
