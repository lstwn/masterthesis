\begin{figure}[tpb]
    \centering

    \begin{subfigure}[b]{\textwidth}
        \centering
        \begin{tabular}{c}
            \begin{lstlisting}[keepspaces]
mvrStore(Key, Value) :- set(RepId, Ctr, Key, Value),
                        isCausallyReady(RepId, Ctr),
                        isLeaf(RepId, Ctr).\end{lstlisting}
        \end{tabular}
        \caption{The ``mvrStore'' rule to optimize.}\label{code:mvr-store-rule}
    \end{subfigure}

    \vspace{1em}

    \begin{subfigure}[b]{0.46\textwidth}
        \centering
        \begin{tikzpicture}
            \small

            \tikzset{
                node distance=38pt and 48pt,
                on grid,
                node/.style={rectangle,minimum width=0pt,minimum height=0pt,align=center,fill=white},
                input/.style={edge},
                edgelabel/.style={midway},
            }

            \node[node] (projection) {\(\pi_{[\mathit{Key}, \mathit{Value}]}\)};
            \node[node,below=of projection,align=center] (selection) {\(\sigma_{
                    \substack{
                        s.RepId = c.RepId \land s.Ctr = c.Ctr \land \\
                        c.RepId = l.RepId \land c.Ctr = l.Ctr
                    }
                }\)};
            \node[node,below=of selection] (cartesian) {\(\times\)};

            % set
            \node[node,below left=of cartesian] (setalias) {\(\rho_{s}\)};
            \node[node,below=of setalias] (setprojection) {\(\pi_{[\mathit{RepId}, \mathit{Ctr}, \mathit{Key}, \mathit{Value}]}\)};
            \node[node,below=of setprojection] (set) {set};

            \node[node,below right=of cartesian] (cartesian2) {\(\times\)};

            % isCausallyReady
            \node[node,below left=of cartesian2] (causalias) {\(\rho_{c}\)};
            \node[node,below=of causalias] (causprojection) {\(\pi_{[\mathit{RepId}, \mathit{Ctr}]}\)};
            \node[node,below=of causprojection] (caus) {isCausallyReady};

            % isLeaf
            \node[node,below right=of cartesian2] (leafalias) {\(\rho_{l}\)};
            \node[node,below=of leafalias] (leafprojection) {\(\pi_{[\mathit{RepId}, \mathit{Ctr}]}\)};
            \node[node,below=of leafprojection] (leaf) {isLeaf};

            \begin{scope}[on background layer]
                \draw[input] (selection) to[] (projection);
                \draw[input] (cartesian) to[] (selection);

                % set
                \draw[input] (setalias) to[] (cartesian);
                \draw[input] (setprojection) to[] (setalias);
                \draw[input] (set) to[] (setprojection);

                \draw[input] (cartesian2) to[] (cartesian);

                % isCausallyReady
                \draw[input] (causalias) to[] (cartesian2);
                \draw[input] (causprojection) to[] (causalias);
                \draw[input] (caus) to[] (causprojection);

                % isLeaf
                \draw[input] (leafalias) to[] (cartesian2);
                \draw[input] (leafprojection) to[] (leafalias);
                \draw[input] (leaf) to[] (leafprojection);
            \end{scope}
        \end{tikzpicture}
        \caption{The naive query plan.}\label{fig:naive-plan}
    \end{subfigure}
    \hspace{1em}
    \begin{subfigure}[b]{0.46\textwidth}
        \centering
        \begin{tikzpicture}
            \small

            \tikzset{
                node distance=38pt and 48pt,
                on grid,
                node/.style={rectangle,minimum width=0pt,minimum height=0pt,align=center,fill=white},
                input/.style={edge},
                edgelabel/.style={midway},
            }

            \node[node,minimum width=55pt] (projection) {\(\pi_{[\mathit{Key}, \mathit{Value}]}\)};
            \node[node,below=of projection,minimum width=55pt] (equijoin) {\(\bowtie_{[\mathit{RepId}, \mathit{Ctr}]}\)};

            \node[node,below left=of equijoin] (set) {set};
            \node[node,below right=of equijoin] (equijoin2) {\(\bowtie_{[\mathit{RepId}, \mathit{Ctr}]}\)};

            \node[node,below left=of equijoin2] (caus) {isCausallyReady};
            \node[node,below right=of equijoin2] (leaf) {isLeaf};

            \begin{scope}[on background layer]
                \draw[input] (equijoin) to[] (projection);
                \draw[input] (set) to[] (equijoin);

                \draw[input] (equijoin2) to[] (equijoin);
                \draw[input] (caus) to[] (equijoin2);
                \draw[input] (leaf) to[] (equijoin2);

                \draw [decorate,decoration={brace,amplitude=4pt},xshift=0pt,yshift=0pt]
                (projection.east) -- (equijoin.east) node [midway,right,xshift=4pt,yshift=-0pt,align=left] {Collapsed into\\ one operator in \ac{IR}};
            \end{scope}
        \end{tikzpicture}
        \caption{The subpar query plan.}\label{fig:optimized-plan}
    \end{subfigure}

    \caption{
        Different query plans for the ``mvrStore'' predicate of \ref{code:mvr-crdt-datalog}.
    }\label{fig:query-opt}
\end{figure}
