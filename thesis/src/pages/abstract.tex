\chapter{\abstractname}

Concurrent writes make it challenging to define \acsp{CRDT} correctly and
complicate the proof of their convergence.
To eliminate an entire class of convergence-related errors,
researchers have proposed defining \acsp{CRDT} in restricted,
\aclp{DSL} that ensure \acsp{CRDT} always converge to the same result across replicas.

This work explores using Datalog --- applied over a monotonically growing set of
operations --- as a \acl{DSL} for defining \acsp{CRDT} with guaranteed convergence.
To avoid performance degradation as the set of operations grows,
I investigate the use of incremental computation via the recent DBSP framework
to evaluate Datalog-based \acs{CRDT}  queries.
I present \acs{CRDT} implementations of a key value store and a list using a custom
Datalog dialect that supports negation and self-recursion.
Their performance is evaluated on a custom-built, incremental query engine based
on relational algebra.
Consequently, this work also encompasses the translation from Datalog programs
into reasonably efficient relational query plans.

This work is an initial step towards a future in which application developers are
empowered to define their own \acsp{CRDT} in a high-level query language without
needing to worry about their concrete implementation nor their convergence.
More broadly, it contributes to the ongoing effort to understand the performance
cost of guaranteed convergence.

\selectlanguage{ngerman}
\chapter{Kurzfassung}

Gleichzeitige Schreibzugriffe erschweren die korrekte Definition von CRDTs und
verkomplizieren den Beweis ihrer Konvergenz.
Um eine ganze Klasse konvergenzbezogener Fehler zu eliminieren,
haben Forscher vorgeschlagen, CRDTs in eingeschränkten, domänenspezifischen Sprachen
zu definieren, die sicherstellen, dass CRDTs auf allen Replikaten stets zum
gleichen Ergebnis konvergieren.

Diese Arbeit untersucht die Verwendung von Datalog --- angewendet auf eine monoton
wachsende Menge von Operationen --- als domänenspezifische Sprache zur Definition
von CRDTs mit garantierter Konvergenz.
Um einer Leistungsverschlechterung bei wachsender Operationsmenge entgegenzuwirken,
wird der Einsatz inkrementeller Berechnung mithilfe des DBSP-Frameworks zur
Auswertung von Datalog-basierten CRDT-Anfragen untersucht.
Es werden CRDT Implementierungen eines Key-Value-Stores und einer Liste vorgestellt,
die eine eigene Datalog Variante nutzen, welche Negation und Selbstrekursion unterstützt.
Die Performance wird mithilfe einer eigens entwickelten, inkrementellen Anfrage-Engine
auf Basis relationaler Algebra evaluiert.
Infolgedessen umfasst diese Arbeit auch die Übersetzung von Datalog Programmen
in effizient ausführbare relationale Anfragepläne.

Diese Arbeit stellt einen ersten Schritt in Richtung einer Zukunft dar,
in der Anwendungsentwickler ihre eigenen CRDTs in einer Abfragesprache
definieren können, ohne sich um deren konkrete Implementierung oder
Konvergenzeigenschaften kümmern zu müssen.
Im weiteren Sinne trägt sie zur laufenden Erforschung der Kosten garantierter
Konvergenz bei.
\selectlanguage{american}
