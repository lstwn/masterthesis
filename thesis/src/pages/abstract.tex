\chapter{\abstractname}

\acsp{CRDT} are data structures which allow replicas to converge to the same
state in the presence of uncoordinated, concurrent writes.
The latter make it challenging to define \acsp{CRDT} correctly and
complicate the proof of their convergence.
To eliminate an entire class of convergence-related errors,
researchers have proposed defining \acsp{CRDT} in restricted,
\aclp{DSL} that ensure \acsp{CRDT} always converge.

This work explores using Datalog --- applied over a monotonically growing set of
operations --- as a \acl{DSL} for defining \acsp{CRDT} with guaranteed convergence.
To avoid performance degradation as the set of operations grows,
I utilize incremental computation via the recent DBSP framework to evaluate
Datalog \acs{CRDT} queries on a custom-built query engine.
Its performance is evaluated using a key-value store and a list \acs{CRDT}.

This work is an initial step towards a future in which application developers are
empowered to define their own \acsp{CRDT} in a high-level query language without
needing to worry about their concrete implementation nor their convergence.
More broadly, it contributes to the ongoing effort to understand the performance
cost of guaranteed convergence.

\selectlanguage{ngerman}
\chapter{Kurzfassung}

\acsp{CRDT} sind Datenstrukturen, die es Replikaten ermöglichen, zum gleichen Zustand
zu konvergieren, trotz unkoordinierter, nebenläufiger Schreibzugriffe.
Letztere erschweren die korrekte Definition von \acsp{CRDT} und
machen den Beweis ihrer Konvergenz kompliziert.
Um eine ganze Klasse konvergenzbezogener Fehler zu eliminieren, haben Forscher
vorgeschlagen, \acsp{CRDT} in eingeschränkten, domänenspezifischen Sprachen
zu definieren, die ihre Konvergenz garantieren.

Diese Arbeit untersucht die Verwendung von Datalog --- angewendet auf eine monoton
wachsende Menge von Operationen --- als domänenspezifische Sprache zur Definition
von \acsp{CRDT} mit garantierter Konvergenz.
Um einer Leistungsverschlechterung bei wachsender Operationsmenge entgegenzuwirken,
wird inkrementelle Berechnung mit Hilfe des DBSP-Frameworks benutzt,
um Datalog-\acs{CRDT}-Abfragen auf einer eigens entwickelten Abfrage-Engine auszuführen.
Die Performance wird mithilfe eines Key-Value-Stores und einer Listen-\acs{CRDT} evaluiert.

Diese Arbeit stellt einen ersten Schritt für eine Zukunft dar,
in der Anwendungsentwickler ihre eigenen \acsp{CRDT} in einer Abfragesprache
definieren können, ohne sich um deren konkrete Implementierung oder
Konvergenzeigenschaften sorgen zu müssen.
Im weiteren Sinne trägt sie zur laufenden Erforschung der Kosten garantierter
Konvergenz bei.
\selectlanguage{american}
