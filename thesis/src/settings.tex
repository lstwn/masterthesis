\PassOptionsToPackage{table,svgnames,dvipsnames}{xcolor}

\usepackage[a-2u]{pdfx} % Generate PDF/A: archival compliant, self-contained pdf.
\usepackage[utf8]{inputenc}
\usepackage[T1]{fontenc}
\usepackage[sc]{mathpazo}
\usepackage[ngerman,american]{babel}
\usepackage[autostyle]{csquotes}
\usepackage[%
	backend=biber,
	url=false,
	style=ieee,
	maxnames=4,
	minnames=3,
	maxbibnames=99,
	giveninits,
	uniquename=init]{biblatex}
\usepackage{graphicx}
\usepackage{scrhack} % Necessary for listings package.
\usepackage{listings}
\usepackage{lstautogobble}
\usepackage{tikz}
\usepackage{pgfplots}
\usepackage{pgfplotstable}
\usepackage{booktabs}
\usepackage[final]{microtype}
\usepackage{caption}
\usepackage[printonlyused]{acronym}
\usepackage{ifthen}
\usepackage{amsmath}
\usepackage{amssymb}
\usepackage{subcaption}
\usepackage[nameinlink,noabbrev]{cleveref}

\hypersetup{
	hidelinks, % Remove colored boxes around references and links.
}

% Without \AtBeginDocument it does not persist.
\AtBeginDocument{\renewcommand{\ref}[1]{\Cref{#1}}}

% for fachschaft_print.pdf
\makeatletter
\if@twoside
	\typeout{TUM-Dev LaTeX-Thesis-Template: twoside}
\else
	\typeout{TUM-Dev LaTeX-Thesis-Template: oneside}
\fi
\makeatother

\addto\extrasamerican{
	\def\lstnumberautorefname{Line}
	\def\chapterautorefname{Chapter}
	\def\sectionautorefname{Section}
	\def\subsectionautorefname{Subsection}
	\def\subsubsectionautorefname{Subsubsection}
}

\addto\extrasngerman{
	\def\lstnumberautorefname{Zeile}
}

% Themes
\ifthenelse{\equal{\detokenize{dark}}{\jobname}}{%
	% Dark theme
	\newcommand{\bg}{black} % background
	\newcommand{\fg}{white} % foreground
	\usepackage[pagecolor=\bg]{pagecolor}
	\color{\fg}
}{%
	% Light theme
	\newcommand{\bg}{white} % background
	\newcommand{\fg}{black} % foreground
}

\bibliography{bibliography}

\setkomafont{disposition}{\normalfont\bfseries} % Use serif font for headings.
\linespread{1.05} % Adjust line spread for mathpazo font.

% Add table of contents to PDF bookmarks.
\BeforeTOCHead[toc]{{\cleardoublepage\pdfbookmark[0]{\contentsname}{toc}}}

% Define TUM corporate design colors.
% Taken from http://portal.mytum.de/corporatedesign/index_print/vorlagen/index_farben
\definecolor{TUMBlue}{HTML}{0065BD}
\definecolor{TUMSecondaryBlue}{HTML}{005293}
\definecolor{TUMSecondaryBlue2}{HTML}{003359}
\definecolor{TUMBlack}{HTML}{000000}
\definecolor{TUMWhite}{HTML}{FFFFFF}
\definecolor{TUMDarkGray}{HTML}{333333}
\definecolor{TUMGray}{HTML}{808080}
\definecolor{TUMLightGray}{HTML}{CCCCC6}
\definecolor{TUMAccentGray}{HTML}{DAD7CB}
\definecolor{TUMAccentOrange}{HTML}{E37222}
\definecolor{TUMAccentGreen}{HTML}{A2AD00}
\definecolor{TUMAccentLightBlue}{HTML}{98C6EA}
\definecolor{TUMAccentBlue}{HTML}{64A0C8}

% Settings for pgfplots.
\pgfplotsset{compat=newest}
\pgfplotsset{
	% For available color names, see http://www.latextemplates.com/svgnames-colors
	cycle list={
	    {TUMBlue},{TUMAccentOrange},{TUMAccentGreen},{TUMSecondaryBlue2},{TUMDarkGray}
	},
	legend style={
        anchor=north west,
        draw=none,
        % legend columns=2,
        cells={anchor=west,align=left},
        font=\footnotesize,
    },
}

% Settings for lstlistings.
\lstset{%
	basicstyle=\ttfamily\small,
	columns=fullflexible,
	autogobble,
	keywordstyle=\bfseries\color{TUMBlue},
	stringstyle=\color{TUMAccentGreen},
	captionpos=b
}

% verbatim does not work here, use some alternative like listing or minted?
\newenvironment{codeblock}{
	\begin{small}
		\begin{verbatim}
}{
    \end{verbatim}
	\end{small}
}
\newcommand{\code}[1]{\texttt{#1}}
\newcommand{\setop}[5][set]{$\mathit{#1}_{#2}^{#3} (#4, #5)$}
\newcommand{\var}[1]{\mathit{#1}}
\newcommand{\deltaI}[1]{\(\Delta I_{\text{#1}}\)}
\newcommand{\deltaO}[1][]{\(\Delta O_{\text{#1}}\)}

\newcommand{\joinop}[1][]{\(\bowtie_{[#1]}\)}
\newcommand{\projop}[1][]{\(\pi_{[#1]}\)}
\newcommand{\selop}[1][]{\(\sigma_{#1}\)}
\newcommand{\antijoinop}[1][]{\(\triangleright_{[#1]}\)}
\newcommand{\cartesianop}[0]{\(\times\)}

% Settings for TikZ.
\usetikzlibrary{
	automata,
	positioning,
	arrows.meta,
	calc, backgrounds, quotes,
	patterns, fit,
	decorations.pathreplacing,
	shapes,
	shapes.geometric,
	tikzmark
}
\tikzset{
op/.style={
		font=\footnotesize,
		state,
		minimum size=56.0pt,
		fill=white,
		scale=0.9,
	},
head/.style={
		op,
		accepting,
	},
edge/.style={
->,
>={Stealth[round]},
},
pred/.style={
		edge,
	},
anchorref/.style={
		edge,
		blue,
		densely dashed,
	},
stepmarker/.style={
		font=\footnotesize,
		fill=black!10,
	},
stepline/.style={
		edge,
		black!70,
		dotted,
		semithick,
		-,
	},
}
